\documentclass[letter, 10pt]{article}
\usepackage[utf8]{inputenc}
\usepackage[spanish]{babel}
\usepackage{amsfonts}
\usepackage{amsmath}
\usepackage[dvips]{graphicx}
\usepackage{url}
\usepackage[top=3cm,bottom=3cm,left=3.5cm,right=3.5cm,footskip=1.5cm,headheight=1.5cm,headsep=.5cm,textheight=3cm]{geometry}
\usepackage[linesnumbered]{algorithm2e}
%%\usepackage[]{algorithm}

\begin{document}
\title{Inteligencia Artificial \\ \begin{Large}Informe Final: Problema alanced Academic Curriculum Problem\end{Large}}
\author{[Nombre autor]}
\date{\today}
\maketitle


%--------------------No borrar esta secci\'on--------------------------------%
\section*{Evaluación}

\begin{tabular}{ll}
Mejoras 1ra Entrega (10 \%): &  \underline{\hspace{2cm}}\\
Codigo Fuente (10 \%): &  \underline{\hspace{2cm}}\\
Representación (15 \%):  & \underline{\hspace{2cm}} \\
Descripción del algoritmo (20 \%):  & \underline{\hspace{2cm}} \\
Experimentos (10 \%):  & \underline{\hspace{2cm}} \\
Resultados (10 \%):  & \underline{\hspace{2cm}} \\
Conclusiones (20 \%): &  \underline{\hspace{2cm}}\\
Bibliografía (5 \%): & \underline{\hspace{2cm}}\\
 &  \\
\textbf{Nota Final (100)}:   & \underline{\hspace{2cm}}
\end{tabular}
%---------------------------------------------------------------------------%

\begin{abstract}
Resumen del informe en no m\'as de 10 l\'ineas.
\end{abstract}

\section{Introducción}

\section{Definición del Problema}

\section{Estado del Arte}

\section{Modelo Matemático}


\section{Representación}
\subsection{Cromosoma}
En la búsqueda de una solución a este problema, se implementó un Algoritmo Genético
cuyo cromosoma corresponde a un arreglo de números enteros. El cromosoma tiene, entonces, la siguiente forma:

$$
	[x_{1},x_{2}, \dots ,x_{n-1},x_{n}]
$$

En esta representación la posición $i$ del arreglo representa al ramo $x_{i}$ del plan académico. El dominio
de cada curso $x_{i}$ sera un conjunto de enteros que van desde 1 hasta el número de periodos que debe contemplar la malla
curricular. En otras palabras, cada valor del arreglo indicara el semestre en el cual se dictará el curso $x_{i}$.

Ejemplificando lo anterior para 

\subsubsection{Ventajas de la representación}
Entre las ventajas de esta representación destacan:
\begin{itemize}
	\item El bajo costo de traducción entre la representación y la solución en lenguaje natural.
	\item Manejar de forma simple la restricción dura: "Todos los ramos deben ser instanciado una sola vez",
		del modelo matemático.
	\item Facilitar el calculo del cumplimiento y violación de las restricciones blandas relacionadas con: los pre-requisitos, créditos y numero de ramos máximos y mínimos.
\end{itemize}

\subsubsection{Desventajas de la representación}
\begin{itemize}
	\item Por si sola, esta representación, es extremadamente sensible a operadores genéticos simples de mutación o cruzamiento, 
	por lo que no es capaz de conservar o xxx las restricciones temporales de pre-requisitos.
\end{itemize}

\subsection{Representaciónes auxiliares}
\subsubsection{Skelenton}
El estudio de este problema revela un alto grado de conexión entre ramos y el periodo en el cual son instanciados heredado de las restricciones
temporales de los pre-requisitos. Lo anterior inspiro la creación de una representación auxiliar, propia de cada problema, que fue llamada "esqueleto" del problema.
Este "esqueleto" corresponde a un arreglo binario donde el curso $x_{i}$ tendrá el valor 1 si es parte de algún árbol de pre-requisitos y
0 si no lo es. De esta manera el arreglo "esqueleto" es una representación directa de los grafos de restricciones y servirá para crear,
de forma primitiva, mejoras a los operadores genéticos en busca de una minimización en la tasa de violación de restricciones blandas referentes a los pre-requisitos.

\subsubsection{Max Period}
Para tratar otras las restricciones nacidas de los pre-requisitos, se utilizó un arreglo de enteros el cual indicaba el resultado de calcular el máximo periodo en el cual un ramo puede ser instanciado. Pasado este periodo
la instancicion de dicho ramo generará irremediablemente soluciones infactibles. Al igual que la representación "esqueleto", "max period" es una representación propia de cada problema y no de cada solución.

\subsubsection{Otras representaciones}
\begin{description}
\item[Pre-requisitos:] Los pre-requisitos fueron representados con una matrix bidimencional con columnas de largo variable, donde cada columna
representa un arreglo entero con el identificador del curso requisito.
\item[Creditos:] Arreglo entero que indica el crédito de cada curso.
\item[Cursos:] Arreglo de string que contiene el nombre de cada curso.

\end{description}


\section{Descripción del algoritmo}
Como se mencionó en la sección anterior, se desarrolló un Algoritmo Genético para buscar mallas curriculares balanceadas.
La implementación de este tipo de meta-heristicas requiere el diseño de procedimientos para la creación
de una población inicial, calculo de la calidad de la solución, procedimientos de operadores de mutación, cruzamiento y elitismo, de procedimientos de evaluación de soluciones, de una rutina para la selección de individuos, y el establecimiento de parámetros propios como el tamaño de la población y el número de generaciones.
A continuación se detallan cada uno de estos elementos.

\subsection{Generación de la población inicial}
Debido al número y grado de conexión entre los cursos y  los periodos con las restricciones, se decidió generar soluciones aleatorias que cumplieran con toda
y cada una de las restricciones del problema. Es decir, cada individuo cumplirá no solo con el mínimo y máximo numero de créditos y ramos por semestre, si no que también asegurará que los pre-requisitos de cada ramo sean instanciados con anterioridad. Esto ultimo se logró filtrando de forma dinámica, en cada periodo, los ramos
potencialmente instanciables, para luego seleccionar al azar uno de ellos. En este proceso  la representación "Max Period" juego un papel importante indicando
tempranamente soluciones infactibles en las cuales no sería posible colocar todos los ramos dentro de una malla. Las soluciones que no lograron alcanzar el nivel de factibilidad óptimo, fueron descartadas. Para las instancias analizadas este procedimiento demostró ser eficiente manteniendo el nivel de soluciones no factibles a niveles bajos. El algoritmo \ref{alg:gennew} presenta el pseudo código de esta rutina.

\begin{algorithm}[H]
\SetLine
\KwIn{bacp 8, 10 o 12 de csplib}
\KwOut{Cromosoma de una solución}
\While{Quedan ramos por instanciar}{
Calcular cantidad ramos por instanciar\;
\ForEach{Periodo $p$}{
Crear vector vacío $ramosCandidatos$ \;
\ForEach{Ramo $r$}{
\If{Ramo $r$ tiene todos los requisitos instanciados}{
$ramosCandidatos$ $\gets$ $r$ \;
}
}
\If{$ramosCandidatos$ != $\emptyset$}{
Elegir ramo $i \in ramosCandidatos$ al azar \;
\If{ramo $i$ en el periodo $p$ no rompe restricciones}
{
\tcc{Esta verificación incluye el asegurarse que todos los periodos cuentan con el mínimo de ramos y créditos esperados, antes de tratar de alcanzar el máximo.}
$Cromosoma[i] = p$
}
}
}
\If{no se instanciaron nuevos ramos}{
Detener y desechar solución\;
}
}
\caption{Generar soluciones}
\label{alg:gennew}
\end{algorithm}

\subsection{Calculo de la calidad de la solución}
Siguiendo el modelo matemático presentado en la sección 4, la calidad de la solución viene dada por la suma de las desviaciones a la media.
A esta desviación se le suman el numero de restricciones violadas por la solución. Una solución de mayor calidad, sera aquella que posea
un menor valor de esta sumatoria. El algoritmo \ref{alg:fit} muestra el pseudocódigo que permite calcular este valor.

\begin{algorithm}[H]
\SetLine
\KwIn{Solución $S$, numero de periodos $P$, sumatoria de créditos de todos los ramos $C$}
\KwOut{Fitness $f$}
$w \gets \frac{C}{P}$
\tcc{Promedio de esperado de créditos por semestre que representa el máximo balance teorico}

$f \gets 0$ \;
\ForEach{Periodo $i$}{
$p_{i} \gets$ calculo de los créditos en periodo $i$ según $S$ \;
$f \gets f + |w - p_{i}|$ \;
}
$g \gets$ número de restricciones insatisfechas \;
$f \gets f * (1+g)$ \;

\KwResult{$f$}

\caption{Fitness}
\label{alg:fit}
\end{algorithm}




\subsection{Operador: Mutación}
El operador unario de mutación consiste en seleccionar un individuo y mover algunos de sus ramos a otro semestre.
Esta rutina afecta a todos los ramos y es gatillada previa evaluación de la tasa de aceptación de mutación.
Adicionalmente se diseñaron características combinables para este operador los cuales buscaban interactuar con las restricciones en distintos grados:

\begin{description}
	\item[Mutación normal:] Que evalua las restricciones de mínimos y máximos créditos antes de operar.
	\item[Mutación ciega: ] La cual no tiene en cuenta las restricciones de mínimos y máximos créditos antes de operar
	\item[Mutación con primera mejora:] La cual intenta operar un determinado numero de veces buscando una mejora en la solución.
	\item[Mutación con esqueleto:] Que opera solo si el ramo no es parte del esqueleto de la solución.
\end{description}

El algoritmo \ref{alg:mut} muestra la versión combinada de todas las características antes mencionadas.


\begin{algorithm}[H]
\SetLine
\KwIn{Solución $S$, tasa de mutación $M$, esqueleto $E$, intentos $I$}
\KwOut{Cromosoma de una solución}
$p \gets$ número al azar entre 0 y 1 \;
\If{$p \leq M$}{
\
\ForEach{Ramo $i$}{
\If{$i \in E$}{
Ignorar y continuar iteración \;
}
\If{$p \leq M$}{
$j \gets$ periodo al azar;

\If{Si el periodo $j$ no esta excedido en créditos}{
$S.Cromosoma[i] = j$ \;
}
\If{Solución no ha mejorado en los últimos $I$ intentos}{
$i--$
}
}
}
\KwResult{Cromosoma de la solución S}
}

\caption{Operador: mutación}
\label{alg:mut}
\end{algorithm}


\subsection{Operador: Cruzamiento}
El operador se define como cruzamiento en un punto del cromosoma. Se crearon dos versiones:
\begin{description}
	\item[Cruzamiento en un punto de ramos:] Aplica el cruzamiento directamente al cromosoma.
	\item[Cruzamiento en un punto de periodos:] Calcula los ramos de cada periodo e intenta intercambiar semestres completos entre las dos soluciones. 
\end{description}

Al igual que la mutación, el cruzamiento posee una variante que toma en cuenta los ramos que pertenecen al esqueleto de la solución.
Dada la gran diferencia entre cada implementación, el algoritmo \ref{alg:cruz} solo muestra el pseudocódigo del cruzamiento de semestres enteros.

\begin{algorithm}[H]
\SetLine
\KwIn{Soluciones $S_{1},S_{2}$, tasa de cruzamiento $C$, esqueleto $E$}
\KwOut{Cromosomas de $Z_{1}$ y $Z_{2}$ de dos soluciones}
$w \gets$ número al azar entre 0 y 1 \;
\If{$w \leq C$}{
$Z_{1} = S_{1}$ \;
$Z_{2} = S_{2}$ \;
periodo $P \gets$ periodo al azar distinto del primer y último periodo \;

\ForEach{Ramo $i$}{
\If{$i \in E$}{
Ignorar y continuar \;
}
\If{$S_{1}.Cromosoma[i] \geq P$ o $S_{2}.Cromosoma[i] \geq P$}{
	$Z_{2}.Cromosoma[i] = S_{1}.Cromosoma[i]$ \; 
	$Z_{1}.Cromosoma[i] = S_{2}.Cromosoma[i]$ \;
}
}
\KwResult{$Z_{1},Z_{2}$}
}

\caption{Operador: Cruzamiento de semestres en un punto}
\label{alg:cruz}
\end{algorithm}



\section{Experimentos}
Se necesita saber como experimentaron, como definieron par\'ametros, como los fueron modificando, cuales 
problemas se trataron, instancias, por que ocuparon esos problemas.

\section{Resultados}
Que fue lo que se logr\'o con la experimentaci\'on, incluir tablas y par\'ametros, gr\'aficos si fuera
posible, lo m\'as explicativo posible.

\section{Conclusiones}
De acuerdo a la introducci\'on que se hizo, entregar afirmaciones RELEVANTES basadas en los experimientos
y sus resultados.

\section{Bibliograf\'ia}
Indicando toda la informaci\'on necesaria de acuerdo al tipo de documento revisado. Las referencias deben ser citadas en el documento.
\bibliographystyle{plain}
\bibliography{Referencias}
\end{document} 

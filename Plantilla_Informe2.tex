\documentclass[letter, 10pt]{article}
\usepackage[utf8]{inputenc}
\usepackage[spanish]{babel}
\usepackage{amsfonts}
\usepackage{amsmath}
\usepackage[dvips]{graphicx}
\usepackage{url}
\usepackage{pgf}
\usepackage[top=3cm,bottom=3cm,left=3.5cm,right=3.5cm,footskip=1.5cm,headheight=1.5cm,headsep=.5cm,textheight=3cm]{geometry}
\usepackage[linesnumbered, boxruled]{algorithm2e}
%%\usepackage[]{algorithm}
\usepackage{graphicx}
\begin{document}
\title{Inteligencia Artificial \\ \begin{Large}Informe Final: Problema alanced Academic Curriculum Problem\end{Large}}
\author{Sergio A. Morales}
\date{\today}
\maketitle


%--------------------No borrar esta secci\'on--------------------------------%
\section*{Evaluación}

\begin{tabular}{ll}
Mejoras 1ra Entrega (10 \%): &  \underline{\hspace{2cm}}\\
Codigo Fuente (10 \%): &  \underline{\hspace{2cm}}\\
Representación (15 \%):  & \underline{\hspace{2cm}} \\
Descripción del algoritmo (20 \%):  & \underline{\hspace{2cm}} \\
Experimentos (10 \%):  & \underline{\hspace{2cm}} \\
Resultados (10 \%):  & \underline{\hspace{2cm}} \\
Conclusiones (20 \%): &  \underline{\hspace{2cm}}\\
Bibliografía (5 \%): & \underline{\hspace{2cm}}\\
 &  \\
\textbf{Nota Final (100)}:   & \underline{\hspace{2cm}}
\end{tabular}
%---------------------------------------------------------------------------%

\begin{abstract}
  Balanced Academic Curriculum Problem (BACP) es un problema de optimización,
  presentado el año 2001, que busca generar curriculums o mallas académicas
  balanceadas, es decir, con periodos académicos de carga homogénea. El presente
  informe contiene el estado del arte y el modelo matemático del problema junto
  con el diseño y los resultados del desarrollo de un Algoritmo Genético
  capaz de tratar este problema.
\end{abstract}

\section{Introducción}
En este informe se presenta el estado del arte de Balanced Academic Curriculum
Problem (BACP) como un recorrido por su historia y los avances más destacables
en su estudio.  BACP tiene como objetivo generar un curriculum o malla académica
para un plan de estudios procurando un balance, en cada periodo, de la carga
académica y la cantidad de ramos. Es interesante notar que este problema fue
presentado a la comunidad científica por un trabajo en el que participó nuestro
profesor Dr. Carlos Castro\cite{castro2001variable} en el año 2001 y el cual
posee como referencia una memoria de un alumno del departamento de informática
del año 1994\cite{vergara}. Este hecho, si bien no es crítico para el análisis
del problema, destaca la calidad del plantel docente de nuestra universidad y la
importancia de los problemas tratados por nuestros compañeros.

Como se verá en este documento el problema ha sido tratado con métodos de
programación Lineal, estrategias de propagación de restricciones, algoritmos
genéticos y meta heurísticas de búsquedas locales entre otros, siendo la
propagación de restricciones (método para tratar problemas CSP) el más
recurrente a lo largo de los años.

El lector también podrá apreciar la evolución que ha tenido este
problema hacia unos nuevos y más generales, los cuales tratan de
recoger las necesidades reales que universidades y casas de estudios
han experimentado a lo largo de los años a la hora de diseñar estos
currículum académicos.

Finalmente se presenta la implementación del autor de una meta
heurística evolutiva, en la cual se buscará generar soluciones de
buena calidad aplicando sistemáticamente operadores de exploración y
explotación a una población de soluciones iniciales. Se tratará además
el diseño y las características de los operadores necesarios para
atacar este tipo problemas, se introducirá una representación adecuada
para este problema y se mostraran experimentos que permitan no solo
sintonizar si no también que comparar los resultados esta implementación con
las ya conocidas hasta el momento.



\section{Definición del Problema}
BACP se define como el problema de elegir el orden de cursos a lo largo
de un cierto número de periodos academicos de tal manera que la carga
total de cada periodo este balanceada. Es decir, una vez que los
contenidos y los cursos son definidos por una casa de estudios, estos
deben ser repartidos a lo largo de periodos de tiempo (que llamaremos
periodos académicos), de tal manera que se respete la precedencia entre
los cursos, se vigile que el número y el esfuerzo destinado en cada
periodo no sean excesivamente altos o bajos para cualquier periodo y que
todos y cada uno de los cursos sean asignados a sólo un periodo
académico en particular.

BACP ha sido considerado un problema interesante no sólo por su obvia
importancia para casas de estudio a lo largo del mundo, sino que también por los
múltiples desafíos que este presenta. BACP se comporta como un problema de
satisfacción de restricciones (CSP) a muchos niveles por lo que su solución a
sido atacada principalmente con métodos para resolver este tipo de
problemas\cite{castro2001variable}. Es posible encontrar una colección de
instancias de prueba para problemas BACP en la biblioteca CSPLib\footnote{CSPLib
  es una biblioteca de casos de prueba para problemas de satisfacción de
  restricciones. Mas información en: http://www.csplib.org/}.

BACP también se puede ver como un problema de tipo empaquetamiento, en donde una
colección finita de objetos con distintos pesos tiene que ser colocados dentro
de bolsas de ciertas capacidades. En esta ocasión el objetivo es reducir el
tamaño de las bolsas y no la cantidad de ellas.

BACP, además, se puede considerar como un problema de
scheduling\cite{pinedo2012scheduling} en donde cada curso se convierte en una
actividad indivisible, los créditos representan el consumo de recursos y los
pre-requisitos se convierten en restricciones temporales.

Finalmente, BACP, es un excelente problema de
balanceo\cite{schaus2007deviation}, los cuales se han convertido en un área
interesante de los problemas de satisfacción de restricciones. Problemas donde el
objetivo es satisfacer a un grupo de clientes, compartir recursos escasos o
definir horarios de consultas para médicos se ven altamente beneficiados al
encontrar soluciones balanceadas.

En cuanto a su complejidad, su similitud con problemas tipo empaquetamiento y en
particular con el problema de encontrar una solución dado un número determinado
de bolsas (periodos), nos indican que Balanced Academic Curriculum Problem es un
problema NP-completo, es decir, un problema duro de tratar, el cual se facilita
en cierta medida con la adición de restricciones globales como los
pre-requisitos de cada curso\cite{monette2007cp}.

La presentación oficial del problema a la comunidad científica fue realizado en
el 2001\cite{castro2001variable} y si bien es reconocido como una muy buena
aproximación de los casos reales, también se ha dejado claro, con el paso de los
años, su incompletitud en ciertos aspectos por no considerar factores de
decisión existentes en la realidad como: las planificaciones de las clases que
un profesor debe dictar en la semana\cite{schaerf1999survey}, la existencia de
múltiples opciones a la hora de definir el contenido de un curriculum, etc. Es
así como se definen:

\begin{itemize}
\item \textbf{Número de periodos}: En los cuales cada curso debe ser dictado en sólo un periodo.
\item \textbf{Carga o crédito académico}: Cada curso tiene asociado una carga
  que representa alguna medida de esfuerzo, importancia o tiempo que un alumno
  deberá invertir en dicho curso.
\item \textbf{Pre-requisitos}: Cada curso poseerá un conjunto de otros cursos
  que debieron ser asignado con anterioridad a él.
\item \textbf{Carga Académica mínima y máxima por periodo}: En cada periodo
  existira un mínimo y un máximo número de créditos que deben ser asignados a
  él.
\item \textbf{Número mínimo y máximo de cursos por periodo}: En cada periodo
  existirá un mínimo y un máximo número de cursos que deben ser asignados a él.
\item \textbf{Curriculum académico}: Se refiere a la colección de cursos y sus
  pre-requisitos que componen los objetos a distribuir.
\end{itemize}


El año 2009 \cite{chiarandini2012balanced} abordó este problema con otra
visión dando a luz a los Generalized BACP. En este tipo de problemas la
idea central de los BACP se mantiene, pero se agregan los siguientes
componente al problema:

\begin{itemize}
\item \textbf{Preferencias de los profesores}: Las cuales se pueden
  tratar como restricciones blandas e indican el periodo sobre el cual
  los profesores preferirían que se dictara el ramo.
\item \textbf{Curricula}: Se define como un conjunto de curriculums
  académicos. Este conjunto reune todas las posibles soluciones óptimas
  que son capaces de responder al problema siendo cada una de ellas una
  opción valida que cumple con el plan de estudios.
\end{itemize}

Siguiendo la visión antes mencionada, \cite{schaerfmodelling} definió un
tercer tipo de BACP: GBACP con cursos heterogéneos (GBACP-HC). En esta
definición el problema incluía cursos que deben ser dictados una vez al
año y a los cuales podrán asistir alumnos que estuvieran cursando
diferentes periodos del plan de estudios.

\section{Estado del Arte}
BACP fue definido oficialmente en el año 2001 por Carlos Castro y Sebastián
Manzano\cite{castro2001variable}. En dicho trabajo el problema se resolvía
determinando si un curso era asignado a un periodo en particular, cuidando
minimizar la carga máxima observada a lo largo de todos los periodos. En dicha
ocasión se presentó un modelo de programción lineal entera (ILP) y un modelo
basado en satisfacción de restricciones (CP).  Para resolver estos modelos se
utilizo lp\_solve\footnote{http://lpsolve.sourceforge.net} y
Oz\footnote{http://itee.uq.edu.au/~smith/objectz.html} respectivamente,
logrando obtener buenos resultados sólo para problemas de mediana envergadura:
solución optima en problemas de 8 periodos, solución no optima para problemas de
10 periodos y ninguna solución, por time-out, para problemas de 12
periodos. Cambiando la heurística que determinaba el orden de asignación de
variables, Castro y Manzano lograron reducir el tiempo de cómputo y mejorar el
rendimiento del algoritmo utilizado. 

Con posterioridad en el año 2002 \cite{hnich2002modelling} trabajó sobre lo
desarrollado por Castro y Manzano. \cite{hnich2002modelling} buscaba generar
modelos híbridos CP e ILP con el objetivo de mejorar el rendimiento.
\cite{hnich2002modelling} hizo notar que los modelos propuestos en
\cite{castro2001variable} eran capaces de manejar correctamente los requisitos
de carga académica pero no lograban establecer de forma eficiente las
restricciones de pre-requisitos entre cursos. Debido a esto último propuso un
nuevo modelo CP que hacia mejor uso de las restricciones globales (para
beneficiar su fácil propagación) y que establecía las restricciones de
pre-requisito de manera más simple y eficiente.  De esta manera, durante su
trabajo, logro combinar su modelo con los ya establecidos para aprovechar las
ventajas de cada uno en cuanto al manejo de las restricciones de carga académica
y de pre-requisitos. Sus implementaciones arrojaron mejores resultados que los
ya conocidos, encontrando soluciones donde antes \cite{castro2001variable} no
logró concluir resultados. También demostró que la combinación de los dos
modelos CP era la forma eficiente de encontrar soluciones óptimas y que el uso
de ILP a su vez, era la única forma de comprobar dichos resultados. El cuadro 4 muestra el resultado en dicha ocasión.


\begin{table}[h]
\centering
\caption{Resultados del trabajo en \cite{hnich2002modelling} en seg.}
\begin{tabular}{|c|c|c|c|}
\hline 
Modelo & 8 periodos & 10 periodos & 12 periodos\tabularnewline
\hline 
\hline 
ILP & 3.45 & 4.23 & 131.30\tabularnewline
\hline 
CP 1(Castro) & 58.52 & $\infty$ & $\infty$\tabularnewline
\hline 
CP 2 (Hnich) & 45.10 & $\infty$ & $\infty$\tabularnewline
\hline 
ILP+CP 2 & 0.81 & 8.44 & 3.05\tabularnewline
\hline 
CP 1+CP 2 & 0.29 & 0.59 & 1.09\tabularnewline
\hline 
\end{tabular}

\end{table}

Como se ha mencionado anteriormente BACP es un problema esencialmente de
satisfacción de restricciones por lo que cualquier avance en la resolución de
esos problemas es atractivamente aplicable en este caso. Es así que en el año
2006 \cite{lambert2006solving} aplicó, y con mucho éxito, técnicas hibridas de
algoritmos evolutivos con métodos de propagación de restricciones para resolver
este tipo de problemas. En \cite{lambert2006solving} buscaban, además, crear un
framework capaz de combinar de manera eficiente estas dos técnicas. Para probar
su trabajo sobre los BACP, establecieron una colección de funciones que tendrían
como objetivo: reducir el dominio (aplicar restricciones globales de carga
académica y de pre-requisitos), dividir el dominio para transformar el problema
en dos sub-problemas sin perder posibles soluciones en el proceso y funciones
que contenían los operadores genéticos propiamente tal.

Sus resultados mostraron que al principio de la búsqueda las funciones
de propagación de restricción se llevaban gran parte de la tarea de búsqueda, y a
medida que el tiempo transcurría la población alcanzaba consistencia local siendo
los operados genéticos quienes terminaban realizando un alto porcentaje de la
tarea de búsqueda de una solución óptima.

Aplicando esta implementación al set de problemas publicados en CSPLib fueron
capaces de encontrar nuevos y mejores resultados, disminuyendo aun más la carga
máxima conocida hasta ese instante. Los cuadros 2, 3 y 4 muestran un resumen de sus resultados.


\begin{table}[h]
\centering
\caption{Resultados del trabajo en  \cite{lambert2006solving} con 8 periodos}
\begin{tabular}{|c|c|c|}
\hline 
Tecnica & Creditos maximos & Tiempo de ejecución\tabularnewline
\hline 
\hline 
ILP & 17 & 1459\tabularnewline
\hline 
GA+CP & 17 & 15.05\tabularnewline
\hline 
\end{tabular}
\end{table}

\begin{table}[h]
\centering
\caption{Resultados del trabajo en \cite{lambert2006solving} con 10 periodos}
\begin{tabular}{|c|c|c|}
\hline 
Tecnica & Creditos maximos & Tiempo de ejecución en segundos\tabularnewline
\hline 
\hline 
ILP & 24 & 1459\tabularnewline
\hline 
GA+CP & 14 & 34.84\tabularnewline
\hline 
\end{tabular}
\end{table}

\begin{table}[h!]
\centering
\caption{Resultados del trabajo en \cite{lambert2006solving} con 12 periodos}
\begin{tabular}{|c|c|c|}
\hline 
Tecnica & Creditos maximos & Tiempo de ejecución en segundos\tabularnewline
\hline 
\hline 
ILP & NA & $\infty$\tabularnewline
\hline 
GA+CP & 18 & 35.20\tabularnewline
\hline 
\end{tabular}
\end{table}



En el año 2007 y durante la 4th International Conference on Integration
of AI and OR Techniques in Constraint Programming for Combinatorial
Optimization Problems, Schaus\cite{schaus2007deviation} introdujo el
concepto de una restricción global de complejidad $O(n)$ (tiempo de
resolución lineal) que servía como mejor criterio de balance en
problemas de balanceo. En dicho trabajo se utilizó los problemas BACP
para ejemplificar como el balance podía ser logrado no al minimizar la
carga máxima por periodos (introducida por \cite{castro2001variable}),
si no que más bien buscando minimizar la desviación de está con respecto
a la carga promedio (carga total/número de periodos). Su trabajo
introdujo 3 nuevos tipos de criterios que servían para asegurar balance
en problemas de CSP:

\begin{itemize}
\item $C(1)=\sum_{x \in X}{|x-m|}$, que representa las sumas de las
  desviaciones del promedio.
\item $C(2)=\sum_{x \in X}{(x-m)^{2}}$, que representa las sumas del cuadrado de las
  desviaciones del promedio.
\item $C(\infty)Max_{x \in X}{|x-m|}$, que representa la maxima
  desviación desde el promedio.
\end{itemize}

Mas tarde ese mismo año, Monette\cite{monette2007cp} rescato el trabajo
realizado el 2002 por Hnich referente a las heurísticas de asignación de
valor y variable, las cuales buscaban asignar la variable con menor
dominio primero, junto con el trabajo realizado por Schaus que entregaba
un nuevo y mejor criterio para resolver problemas que requerían
balanceo, para analizar a fondo el problema BACP. Monette complementó el
estudio del problema no sólo diseñando y creando un generador de
instancia para poder realizar experimentos más completos, si no que
también introduciendo "reglas de dominancia" las cuales tenían en cuenta
no sólo el peso (o carga) de los ramos, si no que también sus
pre-requisitos y los conjunto de pre-requisitos de los que eran
parte. 

Estas reglas tenian como objetivo guiar la búsqueda de manera más
informada pues, según lo observado en \cite{monette2007cp}, los
problemas BACP muchas veces eran simétricos al contar con cursos
diferentes pero con los mismos pre-requisitos y carga académica (es
decir, intercambiables entre sí sin modificar el óptimo). La
introducción de estas reglas permitía instanciar cursos de manera tal
que el óptimo resultante fuera de mejor calidad.

Una de las primeras estrategias mostradas, que incluían estas reglas de
dominancia para guiar las búsquedas, fueron aplicadas sobre métodos
"Branch-and-Bound", en donde en cada iteración se buscaba una solución
con un mejor valor (mejor balance) que la anterior. También, experimento
con métodos de búsquedas locales utilizando Tabu Search para conbinarlas
a Branch-And-Bound. Monette procuraba no sólo buscar un óptimo si no que
también satisfacer las restricciones del problema realizando movimientos
que consistían en elegir la variable y el valor de está que mejor
mejorara la función objetivo. El algoritmo de búsqueda local se detenía
una vez que encontraba el óptimo conocido o después de cierto número de
iteraciones.

Otra alternativa al uso de Branch-and-Bound fue la de agregar una
restricción global en sentido contrario, es decir, forzar a que la
función objetivo sea menor a un valor muy pequeño e ir aumentado dicho
valor hasta encontrar una solución factible. Esta ultima visión ha
mostrado, según Baptiste\cite{baptiste2001constraint}, excelentes
resultados en la solución de problemas de tipo Scheduling.

Los experimentos realizados por Monette además buscaban comparar la
calidad de las soluciones encontradas según los criterios propuestos por
\cite{castro2001variable} y los propuestos por
\cite{schaus2007deviation} sin tener en cuenta el tiempo de
procesamiento. Estos experimentos determinaron que minimizar el criterio
de la desviación cuadrática de \cite{schaus2007deviation} se acercaba de
mejor manera al resto de los métodos pero tendía a ser compleja por lo
que proponía utiliza el criterio de suma de diferencias por ser el
segundo mejor criterio (pero más simple de utilizar). El cuadro 5 muestra los resultados obtenidos con estas técnicas.


\begin{table}[h]
\centering


\begin{tabular}{|c|c|c|c|c|c|}
\hline 
 & Criterio Castro & C(1) & C(2) & C($\infty$) & Promedio\tabularnewline
\hline 
\hline 
Criterio Castro & 0 & 10.62 & 16.53 & 0.06 & 9.07\tabularnewline
\hline 
C(1) & 2.63 & 0 & 6.27 & 0.12 & 3.00\tabularnewline
\hline 
C(2) & 0.28 & 0 & 0 & 0 & 0.09\tabularnewline
\hline 
C(infty) & 10.37 & 18.07 & 23.66 & 0 & 17.36\tabularnewline
\hline 
Promedio & 4.43 & 9.59 & 15.48 & 0.06 & \tabularnewline
\hline 
\end{tabular}
\caption{Comparación entre criterios de balance. Cada fila muestra como un
criterio logra acercarse al optimo del otro segun \cite{monette2007cp}.}
\end{table}


En año 2009 \cite{chiarandini2012balanced} buscó dar mas realismo al modelo y al
problema. Introdujo dos nuevas características: la preferencia de los profesores
(la cuales no siempre podrían ser satisfechas) y la generación de una
``curricula'' como una colección de los curriculum de \cite{castro2001variable},
la cual representaba diferentes opciones a un mismo plan de estudios. En
\cite{chiarandini2012balanced} se bautizó este nuevo problema como Generalized
BACP (GBACP). (Observar que BACP se puede considerar como un caso particular de
GBACP).

Utilizando métodos de ILP y búsqueda local con tabu search dinámico y
simulated annealing, realizó varios experimentos demostrando que este
nuevo problema era aun más complejo de resolver que la instancia
presentada en \cite{castro2001variable}, logrando sólo encontrar
soluciones parciales pero de calidad. Si bien su modelo no cuenta con
una solución optima conocida, mostró como los métodos de búsqueda local
podían, en muchos casos, presentar mejores resultados que los métodos de
ILP en este tipo de problemas. El set de problemas que utilizó en sus
experimento fueron publicadas para su uso posterior y fueron definidas
por el prefijo UD seguido de un entero.

Finalmente, y siguiendo la visión expuesta por \cite{chiarandini2012balanced},
\cite{schaerfmodelling} experimento con problemas GBACP introduciendo el
concepto de cursos heterogéneos, es decir, cursos a los cuales asistirían
alumnos en distintos años del plan de estudios, pero que sólo eran dictados una
vez al año. Si bien esta característica se acerca mucho al comportamiento real
que tienen muchas universidades no siempre son una opción recomendable debido a
los problemas pedagógicos que conlleva dictar cursos a alumnos con distintos
niveles de avance y conocimiento. \cite{schaerfmodelling} bautizó este tipo de
problemas como GBACP-HC.

Para definir el espacio y movimientos de búsqueda,
\cite{schaerfmodelling} experimentó con espacios de búsqueda que
relacionaban cursos y periodos, para así poder definir movimientos que
intercambiaban dos cursos entre periodos o cambiaban el periodo asignado
a un curso; y otro en donde relacionaba un par cursos/curriculum con el año
en que se dictaba el curso. Esto último permitía definir cambios en el
año de un curso por curriculum o el intercambio de un par de estos.

Si bien los resultados de estos experimentos no fueron presentados en dicha
ocación si concluyo dos cosas interesantes respecto a su problema y los
GBACP. Lo primero es que una solución de GBACP era efectivamente una solución
válida de GBACP-HC, y lo segundo fue que en general era conveniente resolver
primero el problema GBACP para luego utilizar ese resultado como punto de inicio
de la nueva búsqueda refinada.

\section{Modelo Matemático}
El modelo matemático propuesto por \cite{castro2001variable} en el año
2001, si bien no ha permanecido totalmente invariante, a sido la base
para la mayoría de los estudios realizados a lo largo de los años,
demostrando así ser un buen modelo al problema. Una de las mejoras que
ha recibido este modelo fue la introducida por \cite{hnich2002modelling}
el cual sustituyó las reglas de pre-requisitos por unas mas simples de
computar y la introducida por \cite{monette2007cp} el cual sustituyó el
criterio de balanceo por la suma de los cuadrados de las desviaciones
desde el promedio.

A continuación se presenta el modelo que mejor, según lo visto en la
sección ``Estado del arte'', define un problema BACP:

Cada curso estará identificado por un entero en el intervalo $[1,n]$. De
manera similar cada periodo estará representado por un entero en el
intervalo $[1,m]$. Con esta información se establecen los siguientes
conjuntos de parámetros:

\begin{itemize}
 \item $\forall i \in [1,n]$, $P_{i}$ será el periodo academico del curso $i$.
 \item $\forall i \in [1,n]$, $w_{i}$ será la carga o los creditos del curso $i$.
 \item $\forall i \in [1,m]$, $L_{i}$ será la carga de cada periodo.
\end{itemize}

A lo anterior se agregan las constantes $a$ y $b$ que contienen el número máximo
y mínimo de cursos que se pueden tomar por periodo y $c$ y $d$ que indican la
mínima y máxima carga académica que se debe tener en cada periodo.

Luego las rescticciones del problema se modelan de la siguiente manera:
\begin{itemize}
\item $\forall (i,j) \in [1,n] \times [1,m]$, $B_{i,j}$ será 1 si y sólo si
  el curso $i$ será dado en el periodo $j$, en caso contrario será 0.
 \item Los pre-requisitos se modelan de la siguiente manera: $\forall (i x j) \in
   requisitos$: $P_{i} < P_{j}$.
 \item $\forall 1 \leq j \leq m$: $L_{j} = \sum_{i=1}^{n}{B_{i,j}*w_i}$,
   representa el cálculo de la carga de un periodo.
 \item $\forall 1 \leq \in n$, $1 \leq j \leq m$: $(P_{i}=j) \Leftrightarrow   (B_{i,j}=1)$.
 \item $\forall j \in [1,m]$: $a \leq \sum_{i=1}^{n}{B_{i,j}} \leq b$, que
   condiciona el mínimo y el máximo de cursos tomados por semestre.
 \item $\forall j \in [1,m]$: $c \leq L_{j} \leq d$, que condiciona el
   mínimo y el máximo de creditos tomados por semestre.
 \item $\forall i \in [1,n]$: $\sum_{j=1}^{m}{B_{ij}=1}$. Lo que
   condiciona que cada curso debe ser instanciado sólo una
   vez.
\end{itemize}

Luego, como se ha mencionado anteriormente, la funcion objetivo busca
minimizar la suma total los cuadrados de la diferencia entre la carga
total y la carga total promedio de cada periodo. A esta sumatoria la
llamaremos $C$ y se calculará de la siguiente manera:

\begin{equation*}
  C = \sum_{i=1}^{m}{(m*L_{i}-w)^{2}}
\end{equation*}

Por lo que la función objetivo será:

\begin{equation*}
  Min\{C\}
\end{equation*}


\section{Implementación de un Algoritmo Genético}
En busca de una implementación capaz de encontrar soluciones de
calidad a este problema se implementó una meta heurística evolutiva
con la siguientes características:

\subsection{Representación: Cromosoma}
En este caso el cromosoma corresponde a un arreglo de números enteros con la siguiente forma:

$$
	[x_{1},x_{2}, \dots ,x_{n-1},x_{n}]
$$

En esta representación la posición $i$ del arreglo representa al ramo $x_{i}$
del plan académico. El dominio de cada curso $x_{i}$ sera un conjunto de enteros
que van desde 1 hasta el número de periodos que debe contemplar la malla
curricular. En otras palabras, cada valor del arreglo indicara el semestre en el
cual se dictará el curso $x_{i}$.

A modo de ejemplo, y utilizando el caso de prueba bacp8 disponible en CSPLib, la solución:
$$
[2,3,2,5,2,1,3,4,4,3,3,5,3,5,5,4,7,7,8,8,6,1,2,5,2,6,7,3,6,6,4,7,8,7,8,5,8,7,7,6,2,1,1,3,8,8]
$$


Representa en la malla de 8 periodos y 46 ramos mostrada en la tabla ~\ref{table:ejemplo}.

\begin{table}[h]
\centering
\label{table:ejemplo}
\caption{BACP 8 - Malla de ejemplo}
\begin{tabular}{|c|l|}
\hline 
Periodo & Ramos\tabularnewline
\hline 
\hline 
1 & mat192 - iei162 - iwn270 - hrw130\tabularnewline
\hline 
2 & dew100 - hcw310 - mat190 - iwn170 - hxwxx2 - iwi365\tabularnewline
\hline 
3 & fis100 - dew101 - mat191 - mat193 - hxwxx1 - iei271 - iei218\tabularnewline
\hline 
4 & fis101 - iwi131 - mat194 - hfw120\tabularnewline
\hline 
5 & iwg101 - fis102 - iei134 - iei141 - mat195 - iei273\tabularnewline
\hline 
6 & iei142 - iei231 - iei281 - iwn261 - iei274\tabularnewline
\hline 
7 & dewxx0 - hcw311 - iei241 - iei233 - iei261 - iei232 - iei262\tabularnewline
\hline 
8 & iei132 - iei133 - iei238 - iei272 - iei161 - iei219 - iei248\tabularnewline
\hline 
\end{tabular}

\end{table}


\subsubsection{Representación: Ventajas}
Entre las ventajas de esta representación destacan:
\begin{itemize}
\item El bajo costo de traducción entre la representación y la solución
  en lenguaje natural.
\item Manejar de forma simple la restricción dura: "Todos los ramos
  deben ser instanciado una sola vez", del modelo matemático.
\item Facilitar el cálculo del cumplimiento y violación de las
  restricciones blandas relacionadas con: los pre-requisitos, créditos y
  número de ramos máximos y mínimos.
\end{itemize}

\subsubsection{Presentación: Desventajas}
\begin{itemize}
	\item Por si sola, esta representación, es extremadamente sensible a
          operadores genéticos simples de mutación o cruzamiento, por lo que no
          es capaz de conservar restricciones temporales de pre-requisitos.
\end{itemize}

\subsection{Representaciones auxiliares}
\subsubsection{Esqueleto}
El estudio de este problema revela un alto grado de conexión entre
ramos y el periodo en el cual son instanciados heredado de las
restricciones temporales de los pre-requisitos: para el problema de 8
periodos la tasa de pertenencia al grafo de requisitos era del 80\%,
en problema de 10 periodos la tasa era del 78\% y para problemas de 12
semestre la tasa alcanzaba el 83\%. Lo anterior inspiro la creación de
una representación auxiliar, propia de cada problema, que fue llamada
"esqueleto" del problema.  Este ``esqueleto'' corresponde a un arreglo
binario donde el curso $x_{i}$ tendrá el valor 1 si es parte de algún
árbol de pre-requisitos y 0 si no lo es. De esta manera el arreglo
``esqueleto'' es una representación directa de los grafos de
restricciones y servirá para crear, de forma primitiva, mejoras a los
operadores genéticos en busca de una minimización en la tasa de
violación de restricciones blandas referentes a los pre-requisitos.

\subsubsection{Max Period}
Para tratar otras de las restricciones nacidas de los pre-requisitos,
se utilizó un arreglo de enteros el cual indicaba el resultado de
calcular el máximo periodo en el cual un ramo puede ser
instanciado. Pasado este periodo la instanciación de dicho ramo
generará irremediablemente soluciones infactibles. Al igual que la
representación esqueleto, ``max period'' es una representación propia
de cada problema y no de cada solución, pues se asociaba a la
profundidad máxima en el que se puede encontrar cada ramo según la
información del caso de prueba.

\subsubsection{Otras representaciones}
\begin{description}
\item[Pre-requisitos:] Los pre-requisitos fueron representados en una matrix
  bidimensional con columnas de largo variable, donde cada columna representa un
  arreglo entero con el identificador del curso requisito.
\item[Créditos:] Arreglo entero que indica el crédito de cada curso.
\item[Cursos:] Arreglo de caracteres que contiene el nombre de cada curso.

\end{description}


\section{Descripción del algoritmo}
Como se mencionó en la sección anterior se desarrolló un Algoritmo
Genético para buscar mallas curriculares balanceadas.  La
implementación de este tipo de meta-heristicas requiere el diseño de
procedimientos para la creación de una población inicial, para el
cálculo de la calidad de las soluciones, procedimientos de operadores
de mutación, cruzamiento y elitismo, de una rutina para la selección
de individuos, y el establecimiento de parámetros propios como el
tamaño de la población y el número de generaciones entre otros.  A
continuación se detallan algunos de estos elementos.

\subsection{Generación de la población inicial}
Debido al número y grado de conexión entre los cursos y los periodos con las
restricciones, se decidió generar soluciones aleatorias que cumplieran con toda
y cada una de las restricciones del problema. Es decir, cada individuo cumplirá
no sólo con el mínimo y máximo número de créditos y ramos por semestre, si no
que también asegurará que los pre-requisitos de cada ramo sean instanciados con
anterioridad. Esto ultimo se logró filtrando de forma dinámica, en cada periodo,
los ramos potencialmente instanciables, para luego seleccionar al azar uno de
ellos. En este proceso la representación ``Max Period'' juego un papel importante
indicando tempranamente soluciones infactibles en las cuales no sería posible
colocar todos los ramos dentro de una malla. Las soluciones que no lograron
alcanzar el nivel de factibilidad óptima, fueron descartadas. Para las
instancias analizadas este procedimiento demostró ser eficiente manteniendo el
nivel de soluciones no factibles a niveles bajos. El algoritmo ~\ref{alg:gennew}
presenta el pseudo código de esta rutina.

\begin{algorithm}[H]
\SetLine
\KwIn{bacp 8, 10 o 12 de csplib}
\KwOut{Cromosoma de una solución}
\While{Quedan ramos por instanciar}{
Calcular cantidad ramos por instanciar\;
\ForEach{Periodo $p$}{
Crear vector vacío $ramosCandidatos$ \;
\ForEach{Ramo $r$}{
\If{Ramo $r$ tiene todos los requisitos instanciados}{
$ramosCandidatos$ $\gets$ $r$ \;
}
}
\If{$ramosCandidatos$ != $\emptyset$}{
Elegir ramo $i \in ramosCandidatos$ al azar \;
\If{ramo $i$ en el periodo $p$ no rompe restricciones}
{
\tcc{Esta verificación incluye el asegurarse que todos los periodos cuentan con el mínimo de ramos y créditos esperados, antes de tratar de alcanzar el máximo.}
$Cromosoma[i] = p$
}
}
}
\If{no se instanciaron nuevos ramos}{
Detener y desechar solución\;
}
}
\caption{Generar soluciones}
\label{alg:gennew}
\end{algorithm}

\subsection{Cálculo de la calidad de la solución}
Siguiendo el modelo matemático presentado en la sección 4, la calidad
de la solución viene dada por la suma de las desviaciones a la media
(según \cite{schaus2007deviation}).  A esta desviación se le suman el
número de restricciones violadas por la solución. Una solución de
mayor calidad, sera aquella que posea un menor valor de esta
sumatoria. El algoritmo ~\ref{alg:fit} muestra el pseudocódigo que
permite calcular este valor.

\begin{algorithm}[H]
\SetLine
\KwIn{Solución $S$, número de periodos $P$, sumatoria de créditos de todos los ramos $C$}
\KwOut{Fitness $f$}
$w \gets \frac{C}{P}$
\tcc{Promedio de esperado de créditos por semestre que representa el máximo balance teórico}

$f \gets 0$ \;
\ForEach{Periodo $i$}{
$p_{i} \gets$ cálculo de los créditos en periodo $i$ según $S$ \;
$f \gets f + |w - p_{i}|$ \;
}
$g \gets$ número de restricciones insatisfechas \;
$f \gets f * (1+g)$ \;

\KwResult{$f$}

\caption{Fitness}
\label{alg:fit}
\end{algorithm}


\subsection{Operador: Mutación}
El operador unitario de mutación consiste en seleccionar un individuo
y mover algunos de sus ramos a otro semestre.  Esta rutina afecta a
todos los ramos y se gatilla previa evaluación de la tasa de
aceptación de mutación.  Adicionalmente se diseñaron características,
combinables entre si, las cuales buscan interactuar con las
restricciones en distintos grados:

\begin{description}
\item[Mutación normal:] Que evalua las restricciones de mínimos y
  máximos créditos antes de operar.
\item[Mutación ciega: ] La cual no tiene en cuenta las restricciones de
  mínimos y máximos créditos antes de operar
\item[Mutación con primera mejora:] La cual intenta operar un
  determinado número de veces buscando una mejora en la solución.
\item[Mutación con esqueleto:] Que opera sólo si el ramo no es parte del
  esqueleto de la solución.
\end{description}

El algoritmo ~\ref{alg:mut} muestra la versión combinada de todas las
características antes mencionadas.


\begin{algorithm}[H]
\SetLine
\KwIn{Solución $S$, tasa de mutación $M$, esqueleto $E$, intentos $I$}
\KwOut{Cromosoma de una solución}
$p \gets$ número al azar entre 0 y 1 \;
\If{$p \leq M$}{
\
\ForEach{Ramo $i$}{
\If{$i \in E$}{
Ignorar y continuar iteración \;
}
\If{$p \leq M$}{
$j \gets$ periodo al azar;

\If{Si el periodo $j$ no esta excedido en créditos}{
$S.Cromosoma[i] = j$ \;
}
\If{Solución no ha mejorado en los últimos $I$ intentos}{
$i--$
}
}
}
\KwResult{Cromosoma de la solución S}
}

\caption{Operador: mutación}
\label{alg:mut}
\end{algorithm}


\subsection{Operador: Cruzamiento}
Este operador se define como un cruzamiento en un punto del cromosoma. Se crearon dos
versiones:
\begin{description}
\item[Cruzamiento en un punto de ramos:] Aplica el cruzamiento directamente al
  cromosoma.
\item[Cruzamiento en un punto de periodos:] Calcula los ramos de cada periodo e
  intenta intercambiar semestres completos entre las dos soluciones.
\end{description}

Al igual que la mutación, el cruzamiento posee una variante inteligente que toma
en cuenta los ramos que pertenecen al esqueleto de la solución.  Dada la gran
diferencia entre cada implementación, el algoritmo ~\ref{alg:cruz} sólo muestra
el pseudocódigo del cruzamiento de semestres enteros.

\begin{algorithm}[H]
\SetLine
\KwIn{Soluciones $S_{1},S_{2}$, tasa de cruzamiento $C$, esqueleto $E$}
\KwOut{Cromosomas de $Z_{1}$ y $Z_{2}$ de dos soluciones}
$w \gets$ número al azar entre 0 y 1 \;
\If{$w \leq C$}{
$Z_{1} = S_{1}$ \;
$Z_{2} = S_{2}$ \;
periodo $P \gets$ periodo al azar distinto del primer y último periodo \;

\ForEach{Ramo $i$}{
\If{$i \in E$}{
Ignorar y continuar \;
}
\If{$S_{1}.Cromosoma[i] \geq P$ o $S_{2}.Cromosoma[i] \geq P$}{
	$Z_{2}.Cromosoma[i] = S_{1}.Cromosoma[i]$ \; 
	$Z_{1}.Cromosoma[i] = S_{2}.Cromosoma[i]$ \;
}
}
\KwResult{$Z_{1},Z_{2}$}
}

\caption{Operador: Cruzamiento de semestres en un punto}
\label{alg:cruz}
\end{algorithm}

\subsection{Elitismo}
Se utilizó un operador elitismo estándar de tamaño 1. Es decir, seleccionar el
mejor elemento de una población y traspasarlo directamente a la siguiente.  El
algoritmo ~\ref{alg:elite} muestra la rutina de este procedimiento.

\begin{algorithm}[H]
\SetLine
\KwIn{Población actual $P$, Población siguiente $O$}

$S \gets$ mejor solución de la población $P$ \;
$O \gets \{S\}$ \;

\caption{Operador: Elitismo}
\label{alg:elite}
\end{algorithm}

\subsection{Algoritmo Genético}
El algoritmo genético se ejecutó con tres pequeñas diferencias: Con elitismo,
sin elitismo y con restart en el elitismo.  El motivo de estos cambios se
explicara en la siguiente sección ~\ref{experimentos}. El algoritmo
~\ref{alg:genetico} presenta la rutina de ejecución del método desarrollado.

\begin{algorithm}[H]
\SetLine
\KwIn{generaciones $G$, tamaño de la población $T$, tasa de mutación $M$, tasa de cruzamiento $C$}

Generar población inicial $P_{a}$ \;
Crear población siguientes $P_{b} \gets \emptyset$ \;
Calcular calidad de soluciones de $P_{a}$ \;
\While{no se hayan alcanzado el número de generaciones $G$}{
\If{se debe aplicar elitismo}{
Operador Elitismo con $P_{a}$ y $P_{b}$ \;
}
\While{$P_{b}$ no tenga el tamaño de población $T$}{
Operador Mutación\;

Operador Cruzamiento\;
}
$P_{a} \gets P_{b}$ \;
Calcular calidad de soluciones de $P_{a}$ \;
$P_{b} \gets \emptyset$ \;
}
\caption{Algoritmo Genético}
\label{alg:genetico}
\end{algorithm}

\section{Implementación}
Para implementar el algoritmo se desarrolló un software en C++ utilizando un
paradigma orientado a objetos. El código de esta implementación se encuentra
disponible en un repositorio GIT en
GitHub \footnote{https://github.com/Aierk/bacp\_ga\_eclipse}. La imagen
~\ref{clases} presenta el diagrama, resumido, de clases de la implementación.

\begin{figure}[h]
  \centering
  \pgfimage[width=\textwidth]{images/ga.pdf}
  \caption{Diagrama de clases}
\end{figure}

\section{Experimentos}
\label{experimentos}
Al igual que en los papers referenciados en la sección Estado del Arte, los
experimentos se realizaron utilizando los casos oficiales almacenados en CSPLib
para mallas de 8, 10 y 12 periodos. A continuación se listaran los experimentos
de forma cualitativa:

\subsection{Determinación de la población y generación}
Estos experimentos tenian como tenían como objetivo observar el
comportamiento del algoritmo con generaciones cercanas a 100 y
poblaciones del ordenes menores.  Utilizando los operadores mutación y
cruzamiento sin mejoras, se observo que el algoritmo realizaba pocas
mejoras si la población era pequeña (8 a 20) y que convergía muy
rápido como para utilizar un número muy alto de generaciones (200 o
más). Los experimentos iniciales también arrojaron que los castigos
por romper restricciones de la función Fitness, utilizada para evaluar
las soluciones, no penalizaban los suficiente a las soluciones
infactibles. Una vez terminados estos experimentos se genero una nueva
función Fitness que castigaba de manera mas acertada las soluciones
infactibles y se concluye que poblaciones de \textbf{30} individuos en
\textbf{100} generaciones permitía a las soluciones evolucionar de manera
adecuada.

\subsection{Determinación de las probabilidades de mutación y cruzamiento}
Durante los experimentos se observó que la tasa de mejora que aplicaban los
operadores cruzamiento y mutación eran extremadamente bajas. Analizando la
evolución de los individuos se noto un rápido empeoramiento en su calidad a
medida que aumentan las probabilidades de los operadores.

Esto impulso la creación de versiones mejoradas de ambos operados (presentados
en la sección anterior) que demostraron disminuir la desviación de las nuevas
poblaciones y de mejorar considerable y rápidamente a la población inicial.

Experimentos posteriores demostraron la necesidad de una mayor exploración en
problemas mas grandes, para así lograr observar un mayor número de cambios entre
generaciones. El resultado de estos experimentos es el siguiente:

\begin{table}[h]
\label{table:tasas}
\caption{Resumen de tasas de mutación y cruzamiento}
\begin{tabular}{|c|l|l|}
\hline 
Problema & Tasa de mutación  & Tasa de cruzamiento \tabularnewline
\hline 
\hline 
bacp 8 & 0.4 (mutación inteligente) & 0.7 (cruzamiento inteligente por periodos) \tabularnewline
\hline 
bacp 10 & 0.6 (mutación inteligente) & 0.7 (cruzamiento inteligente por periodos) \tabularnewline
\hline 
bacp 12 & 0.7 (con mutación simple) & 0.7 (cruzamiento inteligente por periodos) \tabularnewline
\hline 
\end{tabular}
\end{table}

Finalmente los gráficos 1 y 2 muestran la diferencia entre los operadores
simples e inteligentes y su impacto en la evolución de la población.

\begin{center}
% GNUPLOT: LaTeX picture
\setlength{\unitlength}{0.240900pt}
\ifx\plotpoint\undefined\newsavebox{\plotpoint}\fi
\sbox{\plotpoint}{\rule[-0.200pt]{0.400pt}{0.400pt}}%
\begin{picture}(1500,900)(0,0)
\sbox{\plotpoint}{\rule[-0.200pt]{0.400pt}{0.400pt}}%
\put(110.0,131.0){\rule[-0.200pt]{4.818pt}{0.400pt}}
\put(90,131){\makebox(0,0)[r]{ 0}}
\put(1419.0,131.0){\rule[-0.200pt]{4.818pt}{0.400pt}}
\put(110.0,239.0){\rule[-0.200pt]{4.818pt}{0.400pt}}
\put(90,239){\makebox(0,0)[r]{ 10}}
\put(1419.0,239.0){\rule[-0.200pt]{4.818pt}{0.400pt}}
\put(110.0,346.0){\rule[-0.200pt]{4.818pt}{0.400pt}}
\put(90,346){\makebox(0,0)[r]{ 20}}
\put(1419.0,346.0){\rule[-0.200pt]{4.818pt}{0.400pt}}
\put(110.0,454.0){\rule[-0.200pt]{4.818pt}{0.400pt}}
\put(90,454){\makebox(0,0)[r]{ 30}}
\put(1419.0,454.0){\rule[-0.200pt]{4.818pt}{0.400pt}}
\put(110.0,561.0){\rule[-0.200pt]{4.818pt}{0.400pt}}
\put(90,561){\makebox(0,0)[r]{ 40}}
\put(1419.0,561.0){\rule[-0.200pt]{4.818pt}{0.400pt}}
\put(110.0,669.0){\rule[-0.200pt]{4.818pt}{0.400pt}}
\put(90,669){\makebox(0,0)[r]{ 50}}
\put(1419.0,669.0){\rule[-0.200pt]{4.818pt}{0.400pt}}
\put(110.0,776.0){\rule[-0.200pt]{4.818pt}{0.400pt}}
\put(90,776){\makebox(0,0)[r]{ 60}}
\put(1419.0,776.0){\rule[-0.200pt]{4.818pt}{0.400pt}}
\put(110.0,131.0){\rule[-0.200pt]{0.400pt}{4.818pt}}
\put(110,90){\makebox(0,0){ 0}}
\put(110.0,756.0){\rule[-0.200pt]{0.400pt}{4.818pt}}
\put(243.0,131.0){\rule[-0.200pt]{0.400pt}{4.818pt}}
\put(243,90){\makebox(0,0){ 10}}
\put(243.0,756.0){\rule[-0.200pt]{0.400pt}{4.818pt}}
\put(376.0,131.0){\rule[-0.200pt]{0.400pt}{4.818pt}}
\put(376,90){\makebox(0,0){ 20}}
\put(376.0,756.0){\rule[-0.200pt]{0.400pt}{4.818pt}}
\put(509.0,131.0){\rule[-0.200pt]{0.400pt}{4.818pt}}
\put(509,90){\makebox(0,0){ 30}}
\put(509.0,756.0){\rule[-0.200pt]{0.400pt}{4.818pt}}
\put(642.0,131.0){\rule[-0.200pt]{0.400pt}{4.818pt}}
\put(642,90){\makebox(0,0){ 40}}
\put(642.0,756.0){\rule[-0.200pt]{0.400pt}{4.818pt}}
\put(774.0,131.0){\rule[-0.200pt]{0.400pt}{4.818pt}}
\put(774,90){\makebox(0,0){ 50}}
\put(774.0,756.0){\rule[-0.200pt]{0.400pt}{4.818pt}}
\put(907.0,131.0){\rule[-0.200pt]{0.400pt}{4.818pt}}
\put(907,90){\makebox(0,0){ 60}}
\put(907.0,756.0){\rule[-0.200pt]{0.400pt}{4.818pt}}
\put(1040.0,131.0){\rule[-0.200pt]{0.400pt}{4.818pt}}
\put(1040,90){\makebox(0,0){ 70}}
\put(1040.0,756.0){\rule[-0.200pt]{0.400pt}{4.818pt}}
\put(1173.0,131.0){\rule[-0.200pt]{0.400pt}{4.818pt}}
\put(1173,90){\makebox(0,0){ 80}}
\put(1173.0,756.0){\rule[-0.200pt]{0.400pt}{4.818pt}}
\put(1306.0,131.0){\rule[-0.200pt]{0.400pt}{4.818pt}}
\put(1306,90){\makebox(0,0){ 90}}
\put(1306.0,756.0){\rule[-0.200pt]{0.400pt}{4.818pt}}
\put(1439.0,131.0){\rule[-0.200pt]{0.400pt}{4.818pt}}
\put(1439,90){\makebox(0,0){ 100}}
\put(1439.0,756.0){\rule[-0.200pt]{0.400pt}{4.818pt}}
\put(110.0,131.0){\rule[-0.200pt]{0.400pt}{155.380pt}}
\put(110.0,131.0){\rule[-0.200pt]{320.156pt}{0.400pt}}
\put(1439.0,131.0){\rule[-0.200pt]{0.400pt}{155.380pt}}
\put(110.0,776.0){\rule[-0.200pt]{320.156pt}{0.400pt}}
\put(774,29){\makebox(0,0){Generaciones}}
\put(774,838){\makebox(0,0){Gráfico 1: BACP 8 - Mut. y Cruz. Simple (sin elitismo)}}
\sbox{\plotpoint}{\rule[-0.400pt]{0.800pt}{0.800pt}}%
\sbox{\plotpoint}{\rule[-0.200pt]{0.400pt}{0.400pt}}%
\put(1279,736){\makebox(0,0)[r]{Mejor}}
\sbox{\plotpoint}{\rule[-0.400pt]{0.800pt}{0.800pt}}%
\put(1299.0,736.0){\rule[-0.400pt]{24.090pt}{0.800pt}}
\put(123,249){\usebox{\plotpoint}}
\multiput(124.41,249.00)(0.509,0.683){21}{\rule{0.123pt}{1.286pt}}
\multiput(121.34,249.00)(14.000,16.331){2}{\rule{0.800pt}{0.643pt}}
\multiput(138.41,268.00)(0.509,1.857){19}{\rule{0.123pt}{3.031pt}}
\multiput(135.34,268.00)(13.000,39.709){2}{\rule{0.800pt}{1.515pt}}
\multiput(151.41,314.00)(0.509,2.270){19}{\rule{0.123pt}{3.646pt}}
\multiput(148.34,314.00)(13.000,48.432){2}{\rule{0.800pt}{1.823pt}}
\multiput(164.41,370.00)(0.509,3.842){19}{\rule{0.123pt}{5.985pt}}
\multiput(161.34,370.00)(13.000,81.579){2}{\rule{0.800pt}{2.992pt}}
\multiput(176.00,462.08)(0.710,-0.514){13}{\rule{1.320pt}{0.124pt}}
\multiput(176.00,462.34)(11.260,-10.000){2}{\rule{0.660pt}{0.800pt}}
\multiput(191.41,454.00)(0.509,1.278){19}{\rule{0.123pt}{2.169pt}}
\multiput(188.34,454.00)(13.000,27.498){2}{\rule{0.800pt}{1.085pt}}
\multiput(204.41,486.00)(0.509,1.278){19}{\rule{0.123pt}{2.169pt}}
\multiput(201.34,486.00)(13.000,27.498){2}{\rule{0.800pt}{1.085pt}}
\multiput(216.00,519.40)(0.639,0.512){15}{\rule{1.218pt}{0.123pt}}
\multiput(216.00,516.34)(11.472,11.000){2}{\rule{0.609pt}{0.800pt}}
\multiput(231.41,529.00)(0.509,2.270){19}{\rule{0.123pt}{3.646pt}}
\multiput(228.34,529.00)(13.000,48.432){2}{\rule{0.800pt}{1.823pt}}
\multiput(244.41,567.82)(0.509,-2.601){19}{\rule{0.123pt}{4.138pt}}
\multiput(241.34,576.41)(13.000,-55.410){2}{\rule{0.800pt}{2.069pt}}
\multiput(257.41,505.10)(0.509,-2.394){19}{\rule{0.123pt}{3.831pt}}
\multiput(254.34,513.05)(13.000,-51.049){2}{\rule{0.800pt}{1.915pt}}
\multiput(297.41,462.00)(0.509,4.587){19}{\rule{0.123pt}{7.092pt}}
\multiput(294.34,462.00)(13.000,97.280){2}{\rule{0.800pt}{3.546pt}}
\put(269.0,462.0){\rule[-0.400pt]{6.504pt}{0.800pt}}
\multiput(324.41,568.57)(0.509,-0.698){19}{\rule{0.123pt}{1.308pt}}
\multiput(321.34,571.29)(13.000,-15.286){2}{\rule{0.800pt}{0.654pt}}
\multiput(337.41,556.00)(0.509,5.373){19}{\rule{0.123pt}{8.262pt}}
\multiput(334.34,556.00)(13.000,113.853){2}{\rule{0.800pt}{4.131pt}}
\multiput(350.41,673.60)(0.509,-1.980){21}{\rule{0.123pt}{3.229pt}}
\multiput(347.34,680.30)(14.000,-46.299){2}{\rule{0.800pt}{1.614pt}}
\multiput(364.41,617.33)(0.509,-2.518){19}{\rule{0.123pt}{4.015pt}}
\multiput(361.34,625.67)(13.000,-53.666){2}{\rule{0.800pt}{2.008pt}}
\multiput(377.41,572.00)(0.509,1.733){19}{\rule{0.123pt}{2.846pt}}
\multiput(374.34,572.00)(13.000,37.093){2}{\rule{0.800pt}{1.423pt}}
\multiput(389.00,616.40)(0.589,0.512){15}{\rule{1.145pt}{0.123pt}}
\multiput(389.00,613.34)(10.623,11.000){2}{\rule{0.573pt}{0.800pt}}
\multiput(403.41,595.04)(0.509,-4.802){21}{\rule{0.123pt}{7.457pt}}
\multiput(400.34,610.52)(14.000,-111.522){2}{\rule{0.800pt}{3.729pt}}
\multiput(417.41,499.00)(0.509,3.428){19}{\rule{0.123pt}{5.369pt}}
\multiput(414.34,499.00)(13.000,72.856){2}{\rule{0.800pt}{2.685pt}}
\multiput(430.41,583.00)(0.509,3.056){19}{\rule{0.123pt}{4.815pt}}
\multiput(427.34,583.00)(13.000,65.005){2}{\rule{0.800pt}{2.408pt}}
\put(309.0,574.0){\rule[-0.400pt]{3.373pt}{0.800pt}}
\multiput(457.41,651.55)(0.509,-0.864){19}{\rule{0.123pt}{1.554pt}}
\multiput(454.34,654.77)(13.000,-18.775){2}{\rule{0.800pt}{0.777pt}}
\multiput(470.41,607.07)(0.509,-4.504){19}{\rule{0.123pt}{6.969pt}}
\multiput(467.34,621.54)(13.000,-95.535){2}{\rule{0.800pt}{3.485pt}}
\multiput(483.41,526.00)(0.509,4.835){19}{\rule{0.123pt}{7.462pt}}
\multiput(480.34,526.00)(13.000,102.513){2}{\rule{0.800pt}{3.731pt}}
\multiput(495.00,645.41)(0.492,0.509){21}{\rule{1.000pt}{0.123pt}}
\multiput(495.00,642.34)(11.924,14.000){2}{\rule{0.500pt}{0.800pt}}
\multiput(510.41,629.58)(0.509,-4.421){19}{\rule{0.123pt}{6.846pt}}
\multiput(507.34,643.79)(13.000,-93.790){2}{\rule{0.800pt}{3.423pt}}
\multiput(523.41,550.00)(0.509,5.414){19}{\rule{0.123pt}{8.323pt}}
\multiput(520.34,550.00)(13.000,114.725){2}{\rule{0.800pt}{4.162pt}}
\multiput(536.41,663.38)(0.509,-2.819){21}{\rule{0.123pt}{4.486pt}}
\multiput(533.34,672.69)(14.000,-65.690){2}{\rule{0.800pt}{2.243pt}}
\multiput(550.41,599.27)(0.509,-1.071){19}{\rule{0.123pt}{1.862pt}}
\multiput(547.34,603.14)(13.000,-23.136){2}{\rule{0.800pt}{0.931pt}}
\multiput(563.41,574.32)(0.509,-0.740){19}{\rule{0.123pt}{1.369pt}}
\multiput(560.34,577.16)(13.000,-16.158){2}{\rule{0.800pt}{0.685pt}}
\multiput(576.41,561.00)(0.509,1.733){19}{\rule{0.123pt}{2.846pt}}
\multiput(573.34,561.00)(13.000,37.093){2}{\rule{0.800pt}{1.423pt}}
\multiput(588.00,605.40)(0.639,0.512){15}{\rule{1.218pt}{0.123pt}}
\multiput(588.00,602.34)(11.472,11.000){2}{\rule{0.609pt}{0.800pt}}
\put(602,614.34){\rule{3.132pt}{0.800pt}}
\multiput(602.00,613.34)(6.500,2.000){2}{\rule{1.566pt}{0.800pt}}
\multiput(616.41,617.00)(0.509,1.319){19}{\rule{0.123pt}{2.231pt}}
\multiput(613.34,617.00)(13.000,28.370){2}{\rule{0.800pt}{1.115pt}}
\multiput(629.41,643.95)(0.509,-0.798){21}{\rule{0.123pt}{1.457pt}}
\multiput(626.34,646.98)(14.000,-18.976){2}{\rule{0.800pt}{0.729pt}}
\multiput(643.41,628.00)(0.509,1.195){19}{\rule{0.123pt}{2.046pt}}
\multiput(640.34,628.00)(13.000,25.753){2}{\rule{0.800pt}{1.023pt}}
\multiput(656.41,658.00)(0.509,1.278){19}{\rule{0.123pt}{2.169pt}}
\multiput(653.34,658.00)(13.000,27.498){2}{\rule{0.800pt}{1.085pt}}
\multiput(669.41,664.39)(0.509,-3.966){19}{\rule{0.123pt}{6.169pt}}
\multiput(666.34,677.20)(13.000,-84.195){2}{\rule{0.800pt}{3.085pt}}
\multiput(682.41,593.00)(0.509,0.798){21}{\rule{0.123pt}{1.457pt}}
\multiput(679.34,593.00)(14.000,18.976){2}{\rule{0.800pt}{0.729pt}}
\multiput(696.41,615.00)(0.509,2.188){19}{\rule{0.123pt}{3.523pt}}
\multiput(693.34,615.00)(13.000,46.688){2}{\rule{0.800pt}{1.762pt}}
\multiput(708.00,670.40)(0.654,0.514){13}{\rule{1.240pt}{0.124pt}}
\multiput(708.00,667.34)(10.426,10.000){2}{\rule{0.620pt}{0.800pt}}
\multiput(721.00,677.08)(0.920,-0.520){9}{\rule{1.600pt}{0.125pt}}
\multiput(721.00,677.34)(10.679,-8.000){2}{\rule{0.800pt}{0.800pt}}
\put(442.0,658.0){\rule[-0.400pt]{3.373pt}{0.800pt}}
\multiput(749.41,659.19)(0.509,-1.733){19}{\rule{0.123pt}{2.846pt}}
\multiput(746.34,665.09)(13.000,-37.093){2}{\rule{0.800pt}{1.423pt}}
\multiput(761.00,626.06)(1.768,-0.560){3}{\rule{2.280pt}{0.135pt}}
\multiput(761.00,626.34)(8.268,-5.000){2}{\rule{1.140pt}{0.800pt}}
\multiput(774.00,621.08)(0.639,-0.512){15}{\rule{1.218pt}{0.123pt}}
\multiput(774.00,621.34)(11.472,-11.000){2}{\rule{0.609pt}{0.800pt}}
\multiput(789.41,587.92)(0.509,-3.718){19}{\rule{0.123pt}{5.800pt}}
\multiput(786.34,599.96)(13.000,-78.962){2}{\rule{0.800pt}{2.900pt}}
\multiput(802.41,521.00)(0.509,5.497){19}{\rule{0.123pt}{8.446pt}}
\multiput(799.34,521.00)(13.000,116.470){2}{\rule{0.800pt}{4.223pt}}
\multiput(814.00,656.40)(0.920,0.520){9}{\rule{1.600pt}{0.125pt}}
\multiput(814.00,653.34)(10.679,8.000){2}{\rule{0.800pt}{0.800pt}}
\multiput(829.41,641.73)(0.509,-3.263){19}{\rule{0.123pt}{5.123pt}}
\multiput(826.34,652.37)(13.000,-69.367){2}{\rule{0.800pt}{2.562pt}}
\multiput(842.41,583.00)(0.509,1.815){19}{\rule{0.123pt}{2.969pt}}
\multiput(839.34,583.00)(13.000,38.837){2}{\rule{0.800pt}{1.485pt}}
\multiput(855.41,610.57)(0.509,-2.628){21}{\rule{0.123pt}{4.200pt}}
\multiput(852.34,619.28)(14.000,-61.283){2}{\rule{0.800pt}{2.100pt}}
\multiput(869.41,558.00)(0.509,2.312){19}{\rule{0.123pt}{3.708pt}}
\multiput(866.34,558.00)(13.000,49.304){2}{\rule{0.800pt}{1.854pt}}
\multiput(882.41,608.55)(0.509,-0.864){19}{\rule{0.123pt}{1.554pt}}
\multiput(879.34,611.77)(13.000,-18.775){2}{\rule{0.800pt}{0.777pt}}
\multiput(895.41,593.00)(0.509,2.188){19}{\rule{0.123pt}{3.523pt}}
\multiput(892.34,593.00)(13.000,46.688){2}{\rule{0.800pt}{1.762pt}}
\multiput(907.00,645.06)(1.936,-0.560){3}{\rule{2.440pt}{0.135pt}}
\multiput(907.00,645.34)(8.936,-5.000){2}{\rule{1.220pt}{0.800pt}}
\put(921,638.84){\rule{3.132pt}{0.800pt}}
\multiput(921.00,640.34)(6.500,-3.000){2}{\rule{1.566pt}{0.800pt}}
\multiput(935.41,620.29)(0.509,-2.849){19}{\rule{0.123pt}{4.508pt}}
\multiput(932.34,629.64)(13.000,-60.644){2}{\rule{0.800pt}{2.254pt}}
\multiput(948.41,561.29)(0.509,-1.065){21}{\rule{0.123pt}{1.857pt}}
\multiput(945.34,565.15)(14.000,-25.145){2}{\rule{0.800pt}{0.929pt}}
\multiput(962.41,540.00)(0.509,3.387){19}{\rule{0.123pt}{5.308pt}}
\multiput(959.34,540.00)(13.000,71.984){2}{\rule{0.800pt}{2.654pt}}
\multiput(975.41,598.16)(0.509,-3.842){19}{\rule{0.123pt}{5.985pt}}
\multiput(972.34,610.58)(13.000,-81.579){2}{\rule{0.800pt}{2.992pt}}
\multiput(988.41,529.00)(0.509,0.947){19}{\rule{0.123pt}{1.677pt}}
\multiput(985.34,529.00)(13.000,20.519){2}{\rule{0.800pt}{0.838pt}}
\multiput(1001.41,553.00)(0.509,0.569){21}{\rule{0.123pt}{1.114pt}}
\multiput(998.34,553.00)(14.000,13.687){2}{\rule{0.800pt}{0.557pt}}
\multiput(1015.41,558.46)(0.509,-1.526){19}{\rule{0.123pt}{2.538pt}}
\multiput(1012.34,563.73)(13.000,-32.731){2}{\rule{0.800pt}{1.269pt}}
\multiput(1028.41,531.00)(0.509,0.864){19}{\rule{0.123pt}{1.554pt}}
\multiput(1025.34,531.00)(13.000,18.775){2}{\rule{0.800pt}{0.777pt}}
\multiput(1041.41,538.17)(0.509,-2.209){21}{\rule{0.123pt}{3.571pt}}
\multiput(1038.34,545.59)(14.000,-51.587){2}{\rule{0.800pt}{1.786pt}}
\multiput(1055.41,494.00)(0.509,5.828){19}{\rule{0.123pt}{8.938pt}}
\multiput(1052.34,494.00)(13.000,123.448){2}{\rule{0.800pt}{4.469pt}}
\multiput(1068.41,622.14)(0.509,-2.063){19}{\rule{0.123pt}{3.338pt}}
\multiput(1065.34,629.07)(13.000,-44.071){2}{\rule{0.800pt}{1.669pt}}
\multiput(1081.41,578.04)(0.509,-0.947){19}{\rule{0.123pt}{1.677pt}}
\multiput(1078.34,581.52)(13.000,-20.519){2}{\rule{0.800pt}{0.838pt}}
\multiput(1094.41,561.00)(0.509,1.103){21}{\rule{0.123pt}{1.914pt}}
\multiput(1091.34,561.00)(14.000,26.027){2}{\rule{0.800pt}{0.957pt}}
\multiput(1107.00,592.40)(0.847,0.520){9}{\rule{1.500pt}{0.125pt}}
\multiput(1107.00,589.34)(9.887,8.000){2}{\rule{0.750pt}{0.800pt}}
\multiput(1120.00,597.08)(0.847,-0.520){9}{\rule{1.500pt}{0.125pt}}
\multiput(1120.00,597.34)(9.887,-8.000){2}{\rule{0.750pt}{0.800pt}}
\multiput(1134.41,591.00)(0.509,0.645){21}{\rule{0.123pt}{1.229pt}}
\multiput(1131.34,591.00)(14.000,15.450){2}{\rule{0.800pt}{0.614pt}}
\multiput(1148.41,609.00)(0.509,1.526){19}{\rule{0.123pt}{2.538pt}}
\multiput(1145.34,609.00)(13.000,32.731){2}{\rule{0.800pt}{1.269pt}}
\multiput(1161.41,635.95)(0.509,-1.608){19}{\rule{0.123pt}{2.662pt}}
\multiput(1158.34,641.48)(13.000,-34.476){2}{\rule{0.800pt}{1.331pt}}
\multiput(1174.41,607.00)(0.509,1.608){19}{\rule{0.123pt}{2.662pt}}
\multiput(1171.34,607.00)(13.000,34.476){2}{\rule{0.800pt}{1.331pt}}
\multiput(1187.41,625.06)(0.509,-3.353){21}{\rule{0.123pt}{5.286pt}}
\multiput(1184.34,636.03)(14.000,-78.029){2}{\rule{0.800pt}{2.643pt}}
\multiput(1201.41,558.00)(0.509,5.083){19}{\rule{0.123pt}{7.831pt}}
\multiput(1198.34,558.00)(13.000,107.747){2}{\rule{0.800pt}{3.915pt}}
\multiput(1214.41,682.00)(0.509,2.063){19}{\rule{0.123pt}{3.338pt}}
\multiput(1211.34,682.00)(13.000,44.071){2}{\rule{0.800pt}{1.669pt}}
\multiput(1227.41,683.07)(0.509,-7.853){21}{\rule{0.123pt}{12.029pt}}
\multiput(1224.34,708.03)(14.000,-182.034){2}{\rule{0.800pt}{6.014pt}}
\multiput(1241.41,526.00)(0.509,1.402){19}{\rule{0.123pt}{2.354pt}}
\multiput(1238.34,526.00)(13.000,30.114){2}{\rule{0.800pt}{1.177pt}}
\multiput(1254.41,532.07)(0.509,-4.504){19}{\rule{0.123pt}{6.969pt}}
\multiput(1251.34,546.54)(13.000,-95.535){2}{\rule{0.800pt}{3.485pt}}
\multiput(1267.41,451.00)(0.509,2.399){21}{\rule{0.123pt}{3.857pt}}
\multiput(1264.34,451.00)(14.000,55.994){2}{\rule{0.800pt}{1.929pt}}
\multiput(1281.41,515.00)(0.509,5.083){19}{\rule{0.123pt}{7.831pt}}
\multiput(1278.34,515.00)(13.000,107.747){2}{\rule{0.800pt}{3.915pt}}
\multiput(1294.41,609.30)(0.509,-4.628){19}{\rule{0.123pt}{7.154pt}}
\multiput(1291.34,624.15)(13.000,-98.152){2}{\rule{0.800pt}{3.577pt}}
\multiput(1307.41,526.00)(0.509,5.745){19}{\rule{0.123pt}{8.815pt}}
\multiput(1304.34,526.00)(13.000,121.703){2}{\rule{0.800pt}{4.408pt}}
\multiput(1320.41,653.07)(0.509,-1.904){21}{\rule{0.123pt}{3.114pt}}
\multiput(1317.34,659.54)(14.000,-44.536){2}{\rule{0.800pt}{1.557pt}}
\multiput(1334.41,585.30)(0.509,-4.628){19}{\rule{0.123pt}{7.154pt}}
\multiput(1331.34,600.15)(13.000,-98.152){2}{\rule{0.800pt}{3.577pt}}
\multiput(1347.41,502.00)(0.509,1.526){19}{\rule{0.123pt}{2.538pt}}
\multiput(1344.34,502.00)(13.000,32.731){2}{\rule{0.800pt}{1.269pt}}
\multiput(1360.41,540.00)(0.509,1.484){21}{\rule{0.123pt}{2.486pt}}
\multiput(1357.34,540.00)(14.000,34.841){2}{\rule{0.800pt}{1.243pt}}
\multiput(1374.41,580.00)(0.509,1.278){19}{\rule{0.123pt}{2.169pt}}
\multiput(1371.34,580.00)(13.000,27.498){2}{\rule{0.800pt}{1.085pt}}
\multiput(1387.41,597.38)(0.509,-2.188){19}{\rule{0.123pt}{3.523pt}}
\multiput(1384.34,604.69)(13.000,-46.688){2}{\rule{0.800pt}{1.762pt}}
\multiput(1400.41,550.27)(0.509,-1.071){19}{\rule{0.123pt}{1.862pt}}
\multiput(1397.34,554.14)(13.000,-23.136){2}{\rule{0.800pt}{0.931pt}}
\multiput(1413.41,531.00)(0.509,3.582){21}{\rule{0.123pt}{5.629pt}}
\multiput(1410.34,531.00)(14.000,83.318){2}{\rule{0.800pt}{2.814pt}}
\multiput(1427.41,611.38)(0.509,-2.188){19}{\rule{0.123pt}{3.523pt}}
\multiput(1424.34,618.69)(13.000,-46.688){2}{\rule{0.800pt}{1.762pt}}
\put(735.0,671.0){\rule[-0.400pt]{3.132pt}{0.800pt}}
\sbox{\plotpoint}{\rule[-0.200pt]{0.400pt}{0.400pt}}%
\put(1279,695){\makebox(0,0)[r]{Castigos}}
\put(1299.0,695.0){\rule[-0.200pt]{24.090pt}{0.400pt}}
\put(123,131){\usebox{\plotpoint}}
\multiput(123.00,131.58)(0.637,0.492){19}{\rule{0.609pt}{0.118pt}}
\multiput(123.00,130.17)(12.736,11.000){2}{\rule{0.305pt}{0.400pt}}
\multiput(137.00,140.92)(0.590,-0.492){19}{\rule{0.573pt}{0.118pt}}
\multiput(137.00,141.17)(11.811,-11.000){2}{\rule{0.286pt}{0.400pt}}
\multiput(163.58,131.00)(0.493,2.558){23}{\rule{0.119pt}{2.100pt}}
\multiput(162.17,131.00)(13.000,60.641){2}{\rule{0.400pt}{1.050pt}}
\multiput(176.00,194.92)(0.637,-0.492){19}{\rule{0.609pt}{0.118pt}}
\multiput(176.00,195.17)(12.736,-11.000){2}{\rule{0.305pt}{0.400pt}}
\multiput(190.58,185.00)(0.493,1.250){23}{\rule{0.119pt}{1.085pt}}
\multiput(189.17,185.00)(13.000,29.749){2}{\rule{0.400pt}{0.542pt}}
\multiput(203.58,217.00)(0.493,2.122){23}{\rule{0.119pt}{1.762pt}}
\multiput(202.17,217.00)(13.000,50.344){2}{\rule{0.400pt}{0.881pt}}
\multiput(216.00,271.58)(0.637,0.492){19}{\rule{0.609pt}{0.118pt}}
\multiput(216.00,270.17)(12.736,11.000){2}{\rule{0.305pt}{0.400pt}}
\multiput(230.58,282.00)(0.493,1.686){23}{\rule{0.119pt}{1.423pt}}
\multiput(229.17,282.00)(13.000,40.046){2}{\rule{0.400pt}{0.712pt}}
\multiput(243.58,316.28)(0.493,-2.558){23}{\rule{0.119pt}{2.100pt}}
\multiput(242.17,320.64)(13.000,-60.641){2}{\rule{0.400pt}{1.050pt}}
\multiput(256.58,260.00)(0.493,2.955){23}{\rule{0.119pt}{2.408pt}}
\multiput(255.17,260.00)(13.000,70.003){2}{\rule{0.400pt}{1.204pt}}
\put(150.0,131.0){\rule[-0.200pt]{3.132pt}{0.400pt}}
\multiput(323.58,335.00)(0.493,0.853){23}{\rule{0.119pt}{0.777pt}}
\multiput(322.17,335.00)(13.000,20.387){2}{\rule{0.400pt}{0.388pt}}
\multiput(336.58,352.50)(0.493,-1.250){23}{\rule{0.119pt}{1.085pt}}
\multiput(335.17,354.75)(13.000,-29.749){2}{\rule{0.400pt}{0.542pt}}
\multiput(349.58,325.00)(0.494,1.929){25}{\rule{0.119pt}{1.614pt}}
\multiput(348.17,325.00)(14.000,49.649){2}{\rule{0.400pt}{0.807pt}}
\multiput(363.00,376.92)(0.652,-0.491){17}{\rule{0.620pt}{0.118pt}}
\multiput(363.00,377.17)(11.713,-10.000){2}{\rule{0.310pt}{0.400pt}}
\multiput(376.58,362.09)(0.493,-1.686){23}{\rule{0.119pt}{1.423pt}}
\multiput(375.17,365.05)(13.000,-40.046){2}{\rule{0.400pt}{0.712pt}}
\multiput(389.00,325.58)(0.652,0.491){17}{\rule{0.620pt}{0.118pt}}
\multiput(389.00,324.17)(11.713,10.000){2}{\rule{0.310pt}{0.400pt}}
\multiput(402.58,330.79)(0.494,-1.158){25}{\rule{0.119pt}{1.014pt}}
\multiput(401.17,332.89)(14.000,-29.895){2}{\rule{0.400pt}{0.507pt}}
\multiput(416.00,303.58)(0.590,0.492){19}{\rule{0.573pt}{0.118pt}}
\multiput(416.00,302.17)(11.811,11.000){2}{\rule{0.286pt}{0.400pt}}
\multiput(429.58,314.00)(0.493,2.122){23}{\rule{0.119pt}{1.762pt}}
\multiput(428.17,314.00)(13.000,50.344){2}{\rule{0.400pt}{0.881pt}}
\put(269.0,335.0){\rule[-0.200pt]{13.009pt}{0.400pt}}
\multiput(456.58,364.77)(0.493,-0.853){23}{\rule{0.119pt}{0.777pt}}
\multiput(455.17,366.39)(13.000,-20.387){2}{\rule{0.400pt}{0.388pt}}
\multiput(469.58,341.50)(0.493,-1.250){23}{\rule{0.119pt}{1.085pt}}
\multiput(468.17,343.75)(13.000,-29.749){2}{\rule{0.400pt}{0.542pt}}
\multiput(482.58,314.00)(0.493,1.250){23}{\rule{0.119pt}{1.085pt}}
\multiput(481.17,314.00)(13.000,29.749){2}{\rule{0.400pt}{0.542pt}}
\put(442.0,368.0){\rule[-0.200pt]{3.373pt}{0.400pt}}
\multiput(509.58,342.90)(0.493,-0.814){23}{\rule{0.119pt}{0.746pt}}
\multiput(508.17,344.45)(13.000,-19.451){2}{\rule{0.400pt}{0.373pt}}
\multiput(522.58,325.00)(0.493,3.787){23}{\rule{0.119pt}{3.054pt}}
\multiput(521.17,325.00)(13.000,89.662){2}{\rule{0.400pt}{1.527pt}}
\multiput(535.58,416.79)(0.494,-1.158){25}{\rule{0.119pt}{1.014pt}}
\multiput(534.17,418.89)(14.000,-29.895){2}{\rule{0.400pt}{0.507pt}}
\multiput(549.58,380.41)(0.493,-2.519){23}{\rule{0.119pt}{2.069pt}}
\multiput(548.17,384.71)(13.000,-59.705){2}{\rule{0.400pt}{1.035pt}}
\multiput(562.00,323.92)(0.590,-0.492){19}{\rule{0.573pt}{0.118pt}}
\multiput(562.00,324.17)(11.811,-11.000){2}{\rule{0.286pt}{0.400pt}}
\multiput(575.58,314.00)(0.493,2.519){23}{\rule{0.119pt}{2.069pt}}
\multiput(574.17,314.00)(13.000,59.705){2}{\rule{0.400pt}{1.035pt}}
\multiput(588.58,373.79)(0.494,-1.158){25}{\rule{0.119pt}{1.014pt}}
\multiput(587.17,375.89)(14.000,-29.895){2}{\rule{0.400pt}{0.507pt}}
\multiput(602.58,346.00)(0.493,1.250){23}{\rule{0.119pt}{1.085pt}}
\multiput(601.17,346.00)(13.000,29.749){2}{\rule{0.400pt}{0.542pt}}
\multiput(615.00,376.92)(0.652,-0.491){17}{\rule{0.620pt}{0.118pt}}
\multiput(615.00,377.17)(11.713,-10.000){2}{\rule{0.310pt}{0.400pt}}
\multiput(628.58,364.98)(0.494,-0.791){25}{\rule{0.119pt}{0.729pt}}
\multiput(627.17,366.49)(14.000,-20.488){2}{\rule{0.400pt}{0.364pt}}
\multiput(642.58,342.90)(0.493,-0.814){23}{\rule{0.119pt}{0.746pt}}
\multiput(641.17,344.45)(13.000,-19.451){2}{\rule{0.400pt}{0.373pt}}
\multiput(655.00,323.92)(0.590,-0.492){19}{\rule{0.573pt}{0.118pt}}
\multiput(655.00,324.17)(11.811,-11.000){2}{\rule{0.286pt}{0.400pt}}
\multiput(668.00,314.58)(0.590,0.492){19}{\rule{0.573pt}{0.118pt}}
\multiput(668.00,313.17)(11.811,11.000){2}{\rule{0.286pt}{0.400pt}}
\multiput(681.58,325.00)(0.494,0.754){25}{\rule{0.119pt}{0.700pt}}
\multiput(680.17,325.00)(14.000,19.547){2}{\rule{0.400pt}{0.350pt}}
\multiput(695.00,344.92)(0.590,-0.492){19}{\rule{0.573pt}{0.118pt}}
\multiput(695.00,345.17)(11.811,-11.000){2}{\rule{0.286pt}{0.400pt}}
\multiput(708.58,335.00)(0.493,2.122){23}{\rule{0.119pt}{1.762pt}}
\multiput(707.17,335.00)(13.000,50.344){2}{\rule{0.400pt}{0.881pt}}
\multiput(721.58,386.09)(0.494,-0.754){25}{\rule{0.119pt}{0.700pt}}
\multiput(720.17,387.55)(14.000,-19.547){2}{\rule{0.400pt}{0.350pt}}
\multiput(735.58,368.00)(0.493,0.814){23}{\rule{0.119pt}{0.746pt}}
\multiput(734.17,368.00)(13.000,19.451){2}{\rule{0.400pt}{0.373pt}}
\put(495.0,346.0){\rule[-0.200pt]{3.373pt}{0.400pt}}
\multiput(761.58,385.90)(0.493,-0.814){23}{\rule{0.119pt}{0.746pt}}
\multiput(760.17,387.45)(13.000,-19.451){2}{\rule{0.400pt}{0.373pt}}
\multiput(774.00,366.92)(0.637,-0.492){19}{\rule{0.609pt}{0.118pt}}
\multiput(774.00,367.17)(12.736,-11.000){2}{\rule{0.305pt}{0.400pt}}
\multiput(788.58,349.69)(0.493,-2.122){23}{\rule{0.119pt}{1.762pt}}
\multiput(787.17,353.34)(13.000,-50.344){2}{\rule{0.400pt}{0.881pt}}
\multiput(801.58,303.00)(0.493,1.250){23}{\rule{0.119pt}{1.085pt}}
\multiput(800.17,303.00)(13.000,29.749){2}{\rule{0.400pt}{0.542pt}}
\multiput(814.58,330.79)(0.494,-1.158){25}{\rule{0.119pt}{1.014pt}}
\multiput(813.17,332.89)(14.000,-29.895){2}{\rule{0.400pt}{0.507pt}}
\multiput(828.58,303.00)(0.493,2.122){23}{\rule{0.119pt}{1.762pt}}
\multiput(827.17,303.00)(13.000,50.344){2}{\rule{0.400pt}{0.881pt}}
\multiput(841.00,355.92)(0.590,-0.492){19}{\rule{0.573pt}{0.118pt}}
\multiput(841.00,356.17)(11.811,-11.000){2}{\rule{0.286pt}{0.400pt}}
\multiput(854.58,340.48)(0.494,-1.562){25}{\rule{0.119pt}{1.329pt}}
\multiput(853.17,343.24)(14.000,-40.242){2}{\rule{0.400pt}{0.664pt}}
\multiput(868.58,303.00)(0.493,3.391){23}{\rule{0.119pt}{2.746pt}}
\multiput(867.17,303.00)(13.000,80.300){2}{\rule{0.400pt}{1.373pt}}
\multiput(881.58,385.90)(0.493,-0.814){23}{\rule{0.119pt}{0.746pt}}
\multiput(880.17,387.45)(13.000,-19.451){2}{\rule{0.400pt}{0.373pt}}
\multiput(894.58,368.00)(0.493,1.250){23}{\rule{0.119pt}{1.085pt}}
\multiput(893.17,368.00)(13.000,29.749){2}{\rule{0.400pt}{0.542pt}}
\multiput(907.58,396.98)(0.494,-0.791){25}{\rule{0.119pt}{0.729pt}}
\multiput(906.17,398.49)(14.000,-20.488){2}{\rule{0.400pt}{0.364pt}}
\multiput(921.58,369.41)(0.493,-2.519){23}{\rule{0.119pt}{2.069pt}}
\multiput(920.17,373.71)(13.000,-59.705){2}{\rule{0.400pt}{1.035pt}}
\put(748.0,389.0){\rule[-0.200pt]{3.132pt}{0.400pt}}
\multiput(961.58,314.00)(0.493,2.955){23}{\rule{0.119pt}{2.408pt}}
\multiput(960.17,314.00)(13.000,70.003){2}{\rule{0.400pt}{1.204pt}}
\multiput(974.58,383.09)(0.493,-1.686){23}{\rule{0.119pt}{1.423pt}}
\multiput(973.17,386.05)(13.000,-40.046){2}{\rule{0.400pt}{0.712pt}}
\multiput(987.00,346.58)(0.590,0.492){19}{\rule{0.573pt}{0.118pt}}
\multiput(987.00,345.17)(11.811,11.000){2}{\rule{0.286pt}{0.400pt}}
\multiput(1000.58,353.98)(0.494,-0.791){25}{\rule{0.119pt}{0.729pt}}
\multiput(999.17,355.49)(14.000,-20.488){2}{\rule{0.400pt}{0.364pt}}
\put(934.0,314.0){\rule[-0.200pt]{6.504pt}{0.400pt}}
\multiput(1027.58,335.00)(0.493,0.853){23}{\rule{0.119pt}{0.777pt}}
\multiput(1026.17,335.00)(13.000,20.387){2}{\rule{0.400pt}{0.388pt}}
\multiput(1040.58,352.79)(0.494,-1.158){25}{\rule{0.119pt}{1.014pt}}
\multiput(1039.17,354.89)(14.000,-29.895){2}{\rule{0.400pt}{0.507pt}}
\put(1014.0,335.0){\rule[-0.200pt]{3.132pt}{0.400pt}}
\multiput(1067.58,325.00)(0.493,0.814){23}{\rule{0.119pt}{0.746pt}}
\multiput(1066.17,325.00)(13.000,19.451){2}{\rule{0.400pt}{0.373pt}}
\multiput(1080.00,344.92)(0.590,-0.492){19}{\rule{0.573pt}{0.118pt}}
\multiput(1080.00,345.17)(11.811,-11.000){2}{\rule{0.286pt}{0.400pt}}
\multiput(1093.58,335.00)(0.494,0.791){25}{\rule{0.119pt}{0.729pt}}
\multiput(1092.17,335.00)(14.000,20.488){2}{\rule{0.400pt}{0.364pt}}
\multiput(1107.58,357.00)(0.493,1.250){23}{\rule{0.119pt}{1.085pt}}
\multiput(1106.17,357.00)(13.000,29.749){2}{\rule{0.400pt}{0.542pt}}
\multiput(1120.00,389.58)(0.590,0.492){19}{\rule{0.573pt}{0.118pt}}
\multiput(1120.00,388.17)(11.811,11.000){2}{\rule{0.286pt}{0.400pt}}
\multiput(1133.00,398.92)(0.637,-0.492){19}{\rule{0.609pt}{0.118pt}}
\multiput(1133.00,399.17)(12.736,-11.000){2}{\rule{0.305pt}{0.400pt}}
\multiput(1147.00,389.58)(0.590,0.492){19}{\rule{0.573pt}{0.118pt}}
\multiput(1147.00,388.17)(11.811,11.000){2}{\rule{0.286pt}{0.400pt}}
\multiput(1160.00,398.92)(0.590,-0.492){19}{\rule{0.573pt}{0.118pt}}
\multiput(1160.00,399.17)(11.811,-11.000){2}{\rule{0.286pt}{0.400pt}}
\multiput(1173.00,387.92)(0.590,-0.492){19}{\rule{0.573pt}{0.118pt}}
\multiput(1173.00,388.17)(11.811,-11.000){2}{\rule{0.286pt}{0.400pt}}
\multiput(1186.58,368.69)(0.494,-2.737){25}{\rule{0.119pt}{2.243pt}}
\multiput(1185.17,373.34)(14.000,-70.345){2}{\rule{0.400pt}{1.121pt}}
\multiput(1200.58,303.00)(0.493,2.122){23}{\rule{0.119pt}{1.762pt}}
\multiput(1199.17,303.00)(13.000,50.344){2}{\rule{0.400pt}{0.881pt}}
\put(1054.0,325.0){\rule[-0.200pt]{3.132pt}{0.400pt}}
\multiput(1226.58,351.48)(0.494,-1.562){25}{\rule{0.119pt}{1.329pt}}
\multiput(1225.17,354.24)(14.000,-40.242){2}{\rule{0.400pt}{0.664pt}}
\put(1213.0,357.0){\rule[-0.200pt]{3.132pt}{0.400pt}}
\multiput(1253.00,314.58)(0.590,0.492){19}{\rule{0.573pt}{0.118pt}}
\multiput(1253.00,313.17)(11.811,11.000){2}{\rule{0.286pt}{0.400pt}}
\put(1240.0,314.0){\rule[-0.200pt]{3.132pt}{0.400pt}}
\multiput(1280.58,325.00)(0.493,1.250){23}{\rule{0.119pt}{1.085pt}}
\multiput(1279.17,325.00)(13.000,29.749){2}{\rule{0.400pt}{0.542pt}}
\multiput(1293.58,351.09)(0.493,-1.686){23}{\rule{0.119pt}{1.423pt}}
\multiput(1292.17,354.05)(13.000,-40.046){2}{\rule{0.400pt}{0.712pt}}
\multiput(1306.58,314.00)(0.493,1.250){23}{\rule{0.119pt}{1.085pt}}
\multiput(1305.17,314.00)(13.000,29.749){2}{\rule{0.400pt}{0.542pt}}
\multiput(1319.58,346.00)(0.494,0.791){25}{\rule{0.119pt}{0.729pt}}
\multiput(1318.17,346.00)(14.000,20.488){2}{\rule{0.400pt}{0.364pt}}
\put(1266.0,325.0){\rule[-0.200pt]{3.373pt}{0.400pt}}
\multiput(1346.58,360.69)(0.493,-2.122){23}{\rule{0.119pt}{1.762pt}}
\multiput(1345.17,364.34)(13.000,-50.344){2}{\rule{0.400pt}{0.881pt}}
\multiput(1359.00,314.58)(0.637,0.492){19}{\rule{0.609pt}{0.118pt}}
\multiput(1359.00,313.17)(12.736,11.000){2}{\rule{0.305pt}{0.400pt}}
\multiput(1373.00,325.58)(0.652,0.491){17}{\rule{0.620pt}{0.118pt}}
\multiput(1373.00,324.17)(11.713,10.000){2}{\rule{0.310pt}{0.400pt}}
\multiput(1386.58,335.00)(0.493,1.290){23}{\rule{0.119pt}{1.115pt}}
\multiput(1385.17,335.00)(13.000,30.685){2}{\rule{0.400pt}{0.558pt}}
\multiput(1399.00,366.92)(0.590,-0.492){19}{\rule{0.573pt}{0.118pt}}
\multiput(1399.00,367.17)(11.811,-11.000){2}{\rule{0.286pt}{0.400pt}}
\put(1333.0,368.0){\rule[-0.200pt]{3.132pt}{0.400pt}}
\multiput(1426.58,349.69)(0.493,-2.122){23}{\rule{0.119pt}{1.762pt}}
\multiput(1425.17,353.34)(13.000,-50.344){2}{\rule{0.400pt}{0.881pt}}
\put(1412.0,357.0){\rule[-0.200pt]{3.373pt}{0.400pt}}
\put(1279,654){\makebox(0,0)[r]{Max. carga}}
\multiput(1299,654)(20.756,0.000){5}{\usebox{\plotpoint}}
\put(1399,654){\usebox{\plotpoint}}
\put(123,335){\usebox{\plotpoint}}
\put(123.00,335.00){\usebox{\plotpoint}}
\put(139.25,347.91){\usebox{\plotpoint}}
\put(156.68,357.00){\usebox{\plotpoint}}
\put(174.02,366.32){\usebox{\plotpoint}}
\put(194.16,368.00){\usebox{\plotpoint}}
\put(209.06,357.74){\usebox{\plotpoint}}
\multiput(216,346)(8.319,19.015){2}{\usebox{\plotpoint}}
\put(240.86,369.65){\usebox{\plotpoint}}
\put(249.79,384.73){\usebox{\plotpoint}}
\multiput(256,400)(4.070,-20.352){3}{\usebox{\plotpoint}}
\put(269.25,335.00){\usebox{\plotpoint}}
\put(290.00,335.00){\usebox{\plotpoint}}
\multiput(296,335)(4.858,20.179){2}{\usebox{\plotpoint}}
\put(309.73,389.00){\usebox{\plotpoint}}
\multiput(323,389)(7.812,-19.229){2}{\usebox{\plotpoint}}
\put(347.03,366.34){\usebox{\plotpoint}}
\put(366.19,365.30){\usebox{\plotpoint}}
\multiput(376,357)(7.812,19.229){2}{\usebox{\plotpoint}}
\put(400.36,379.39){\usebox{\plotpoint}}
\put(416.80,389.00){\usebox{\plotpoint}}
\put(435.53,383.47){\usebox{\plotpoint}}
\put(454.28,378.00){\usebox{\plotpoint}}
\put(472.18,372.87){\usebox{\plotpoint}}
\multiput(482,357)(7.812,19.229){2}{\usebox{\plotpoint}}
\put(502.13,383.40){\usebox{\plotpoint}}
\multiput(509,378)(7.812,-19.229){2}{\usebox{\plotpoint}}
\multiput(522,346)(6.006,19.867){2}{\usebox{\plotpoint}}
\put(547.25,379.38){\usebox{\plotpoint}}
\multiput(562,368)(7.812,19.229){2}{\usebox{\plotpoint}}
\multiput(575,400)(6.006,-19.867){2}{\usebox{\plotpoint}}
\put(591.15,361.73){\usebox{\plotpoint}}
\put(602.95,377.27){\usebox{\plotpoint}}
\multiput(615,368)(6.006,19.867){2}{\usebox{\plotpoint}}
\put(630.14,411.00){\usebox{\plotpoint}}
\multiput(642,411)(7.607,-19.311){2}{\usebox{\plotpoint}}
\put(660.48,391.90){\usebox{\plotpoint}}
\multiput(668,411)(4.858,-20.179){3}{\usebox{\plotpoint}}
\multiput(681,357)(8.319,19.015){2}{\usebox{\plotpoint}}
\put(701.89,400.65){\usebox{\plotpoint}}
\put(712.45,403.48){\usebox{\plotpoint}}
\put(724.10,391.43){\usebox{\plotpoint}}
\put(741.89,400.00){\usebox{\plotpoint}}
\put(762.26,398.94){\usebox{\plotpoint}}
\put(776.98,384.53){\usebox{\plotpoint}}
\put(788.68,367.43){\usebox{\plotpoint}}
\multiput(801,357)(3.545,20.451){4}{\usebox{\plotpoint}}
\put(823.06,424.88){\usebox{\plotpoint}}
\put(833.45,407.59){\usebox{\plotpoint}}
\put(841.69,389.00){\usebox{\plotpoint}}
\put(860.64,394.21){\usebox{\plotpoint}}
\put(876.70,392.64){\usebox{\plotpoint}}
\put(888.96,376.15){\usebox{\plotpoint}}
\put(902.53,360.78){\usebox{\plotpoint}}
\put(915.27,369.40){\usebox{\plotpoint}}
\put(926.30,386.97){\usebox{\plotpoint}}
\put(936.86,395.16){\usebox{\plotpoint}}
\put(947.65,378.51){\usebox{\plotpoint}}
\multiput(961,389)(10.925,-17.648){2}{\usebox{\plotpoint}}
\put(984.47,350.28){\usebox{\plotpoint}}
\multiput(987,346)(4.858,20.179){2}{\usebox{\plotpoint}}
\multiput(1000,400)(8.319,-19.015){2}{\usebox{\plotpoint}}
\put(1022.34,368.00){\usebox{\plotpoint}}
\put(1042.52,369.80){\usebox{\plotpoint}}
\multiput(1054,378)(7.607,19.311){2}{\usebox{\plotpoint}}
\multiput(1067,411)(4.858,-20.179){3}{\usebox{\plotpoint}}
\put(1094.87,347.47){\usebox{\plotpoint}}
\multiput(1107,357)(6.006,19.867){2}{\usebox{\plotpoint}}
\multiput(1120,400)(6.006,-19.867){3}{\usebox{\plotpoint}}
\multiput(1147,346)(10.559,17.869){2}{\usebox{\plotpoint}}
\multiput(1173,368)(6.006,19.867){2}{\usebox{\plotpoint}}
\multiput(1186,411)(5.209,-20.091){3}{\usebox{\plotpoint}}
\put(1203.94,363.37){\usebox{\plotpoint}}
\multiput(1213,378)(7.607,19.311){2}{\usebox{\plotpoint}}
\multiput(1226,411)(4.370,-20.290){3}{\usebox{\plotpoint}}
\multiput(1240,346)(4.858,20.179){3}{\usebox{\plotpoint}}
\multiput(1253,400)(4.858,-20.179){3}{\usebox{\plotpoint}}
\multiput(1280,346)(7.812,19.229){2}{\usebox{\plotpoint}}
\multiput(1293,378)(7.812,-19.229){2}{\usebox{\plotpoint}}
\put(1320.36,355.93){\usebox{\plotpoint}}
\put(1335.46,342.03){\usebox{\plotpoint}}
\multiput(1346,325)(10.925,17.648){2}{\usebox{\plotpoint}}
\put(1366.03,362.07){\usebox{\plotpoint}}
\put(1375.67,375.94){\usebox{\plotpoint}}
\put(1389.93,361.35){\usebox{\plotpoint}}
\multiput(1399,346)(6.006,19.867){3}{\usebox{\plotpoint}}
\put(1420.13,370.43){\usebox{\plotpoint}}
\put(1430.66,360.94){\usebox{\plotpoint}}
\put(1439,368){\usebox{\plotpoint}}
\put(110.0,131.0){\rule[-0.200pt]{0.400pt}{155.380pt}}
\put(110.0,131.0){\rule[-0.200pt]{320.156pt}{0.400pt}}
\put(1439.0,131.0){\rule[-0.200pt]{0.400pt}{155.380pt}}
\put(110.0,776.0){\rule[-0.200pt]{320.156pt}{0.400pt}}
\end{picture}

\end{center}



\begin{center}
% GNUPLOT: LaTeX picture
\setlength{\unitlength}{0.240900pt}
\ifx\plotpoint\undefined\newsavebox{\plotpoint}\fi
\sbox{\plotpoint}{\rule[-0.200pt]{0.400pt}{0.400pt}}%
\begin{picture}(1500,900)(0,0)
\sbox{\plotpoint}{\rule[-0.200pt]{0.400pt}{0.400pt}}%
\put(110.0,131.0){\rule[-0.200pt]{4.818pt}{0.400pt}}
\put(90,131){\makebox(0,0)[r]{ 0}}
\put(1419.0,131.0){\rule[-0.200pt]{4.818pt}{0.400pt}}
\put(110.0,260.0){\rule[-0.200pt]{4.818pt}{0.400pt}}
\put(90,260){\makebox(0,0)[r]{ 5}}
\put(1419.0,260.0){\rule[-0.200pt]{4.818pt}{0.400pt}}
\put(110.0,389.0){\rule[-0.200pt]{4.818pt}{0.400pt}}
\put(90,389){\makebox(0,0)[r]{ 10}}
\put(1419.0,389.0){\rule[-0.200pt]{4.818pt}{0.400pt}}
\put(110.0,518.0){\rule[-0.200pt]{4.818pt}{0.400pt}}
\put(90,518){\makebox(0,0)[r]{ 15}}
\put(1419.0,518.0){\rule[-0.200pt]{4.818pt}{0.400pt}}
\put(110.0,647.0){\rule[-0.200pt]{4.818pt}{0.400pt}}
\put(90,647){\makebox(0,0)[r]{ 20}}
\put(1419.0,647.0){\rule[-0.200pt]{4.818pt}{0.400pt}}
\put(110.0,776.0){\rule[-0.200pt]{4.818pt}{0.400pt}}
\put(90,776){\makebox(0,0)[r]{ 25}}
\put(1419.0,776.0){\rule[-0.200pt]{4.818pt}{0.400pt}}
\put(110.0,131.0){\rule[-0.200pt]{0.400pt}{4.818pt}}
\put(110,90){\makebox(0,0){ 0}}
\put(110.0,756.0){\rule[-0.200pt]{0.400pt}{4.818pt}}
\put(243.0,131.0){\rule[-0.200pt]{0.400pt}{4.818pt}}
\put(243,90){\makebox(0,0){ 10}}
\put(243.0,756.0){\rule[-0.200pt]{0.400pt}{4.818pt}}
\put(376.0,131.0){\rule[-0.200pt]{0.400pt}{4.818pt}}
\put(376,90){\makebox(0,0){ 20}}
\put(376.0,756.0){\rule[-0.200pt]{0.400pt}{4.818pt}}
\put(509.0,131.0){\rule[-0.200pt]{0.400pt}{4.818pt}}
\put(509,90){\makebox(0,0){ 30}}
\put(509.0,756.0){\rule[-0.200pt]{0.400pt}{4.818pt}}
\put(642.0,131.0){\rule[-0.200pt]{0.400pt}{4.818pt}}
\put(642,90){\makebox(0,0){ 40}}
\put(642.0,756.0){\rule[-0.200pt]{0.400pt}{4.818pt}}
\put(774.0,131.0){\rule[-0.200pt]{0.400pt}{4.818pt}}
\put(774,90){\makebox(0,0){ 50}}
\put(774.0,756.0){\rule[-0.200pt]{0.400pt}{4.818pt}}
\put(907.0,131.0){\rule[-0.200pt]{0.400pt}{4.818pt}}
\put(907,90){\makebox(0,0){ 60}}
\put(907.0,756.0){\rule[-0.200pt]{0.400pt}{4.818pt}}
\put(1040.0,131.0){\rule[-0.200pt]{0.400pt}{4.818pt}}
\put(1040,90){\makebox(0,0){ 70}}
\put(1040.0,756.0){\rule[-0.200pt]{0.400pt}{4.818pt}}
\put(1173.0,131.0){\rule[-0.200pt]{0.400pt}{4.818pt}}
\put(1173,90){\makebox(0,0){ 80}}
\put(1173.0,756.0){\rule[-0.200pt]{0.400pt}{4.818pt}}
\put(1306.0,131.0){\rule[-0.200pt]{0.400pt}{4.818pt}}
\put(1306,90){\makebox(0,0){ 90}}
\put(1306.0,756.0){\rule[-0.200pt]{0.400pt}{4.818pt}}
\put(1439.0,131.0){\rule[-0.200pt]{0.400pt}{4.818pt}}
\put(1439,90){\makebox(0,0){ 100}}
\put(1439.0,756.0){\rule[-0.200pt]{0.400pt}{4.818pt}}
\put(110.0,131.0){\rule[-0.200pt]{0.400pt}{155.380pt}}
\put(110.0,131.0){\rule[-0.200pt]{320.156pt}{0.400pt}}
\put(1439.0,131.0){\rule[-0.200pt]{0.400pt}{155.380pt}}
\put(110.0,776.0){\rule[-0.200pt]{320.156pt}{0.400pt}}
\put(774,29){\makebox(0,0){Generaciones}}
\put(774,838){\makebox(0,0){BACP 8 - Mut. y Cruz. Inteligente (sin elitismo)}}
\sbox{\plotpoint}{\rule[-0.400pt]{0.800pt}{0.800pt}}%
\sbox{\plotpoint}{\rule[-0.200pt]{0.400pt}{0.400pt}}%
\put(1279,736){\makebox(0,0)[r]{Mejor}}
\sbox{\plotpoint}{\rule[-0.400pt]{0.800pt}{0.800pt}}%
\put(1299.0,736.0){\rule[-0.400pt]{24.090pt}{0.800pt}}
\put(123,434){\usebox{\plotpoint}}
\multiput(124.41,428.66)(0.509,-0.683){21}{\rule{0.123pt}{1.286pt}}
\multiput(121.34,431.33)(14.000,-16.331){2}{\rule{0.800pt}{0.643pt}}
\multiput(151.41,415.00)(0.509,2.063){19}{\rule{0.123pt}{3.338pt}}
\multiput(148.34,415.00)(13.000,44.071){2}{\rule{0.800pt}{1.669pt}}
\put(137.0,415.0){\rule[-0.400pt]{3.132pt}{0.800pt}}
\multiput(177.41,433.15)(0.509,-5.107){21}{\rule{0.123pt}{7.914pt}}
\multiput(174.34,449.57)(14.000,-118.574){2}{\rule{0.800pt}{3.957pt}}
\put(163.0,466.0){\rule[-0.400pt]{3.132pt}{0.800pt}}
\multiput(244.41,331.00)(0.509,5.538){19}{\rule{0.123pt}{8.508pt}}
\multiput(241.34,331.00)(13.000,117.342){2}{\rule{0.800pt}{4.254pt}}
\multiput(257.41,430.68)(0.509,-5.538){19}{\rule{0.123pt}{8.508pt}}
\multiput(254.34,448.34)(13.000,-117.342){2}{\rule{0.800pt}{4.254pt}}
\put(190.0,331.0){\rule[-0.400pt]{12.768pt}{0.800pt}}
\multiput(297.41,331.00)(0.509,4.752){19}{\rule{0.123pt}{7.338pt}}
\multiput(294.34,331.00)(13.000,100.769){2}{\rule{0.800pt}{3.669pt}}
\multiput(310.41,418.65)(0.509,-4.383){21}{\rule{0.123pt}{6.829pt}}
\multiput(307.34,432.83)(14.000,-101.827){2}{\rule{0.800pt}{3.414pt}}
\put(269.0,331.0){\rule[-0.400pt]{6.504pt}{0.800pt}}
\multiput(364.41,331.00)(0.509,4.752){19}{\rule{0.123pt}{7.338pt}}
\multiput(361.34,331.00)(13.000,100.769){2}{\rule{0.800pt}{3.669pt}}
\put(323.0,331.0){\rule[-0.400pt]{9.636pt}{0.800pt}}
\multiput(417.41,447.00)(0.509,0.740){19}{\rule{0.123pt}{1.369pt}}
\multiput(414.34,447.00)(13.000,16.158){2}{\rule{0.800pt}{0.685pt}}
\put(376.0,447.0){\rule[-0.400pt]{9.636pt}{0.800pt}}
\multiput(443.41,460.66)(0.509,-0.683){21}{\rule{0.123pt}{1.286pt}}
\multiput(440.34,463.33)(14.000,-16.331){2}{\rule{0.800pt}{0.643pt}}
\put(429.0,466.0){\rule[-0.400pt]{3.132pt}{0.800pt}}
\multiput(470.41,447.00)(0.509,0.740){19}{\rule{0.123pt}{1.369pt}}
\multiput(467.34,447.00)(13.000,16.158){2}{\rule{0.800pt}{0.685pt}}
\multiput(483.41,455.46)(0.509,-1.526){19}{\rule{0.123pt}{2.538pt}}
\multiput(480.34,460.73)(13.000,-32.731){2}{\rule{0.800pt}{1.269pt}}
\multiput(496.41,428.00)(0.509,0.683){21}{\rule{0.123pt}{1.286pt}}
\multiput(493.34,428.00)(14.000,16.331){2}{\rule{0.800pt}{0.643pt}}
\put(456.0,447.0){\rule[-0.400pt]{3.132pt}{0.800pt}}
\multiput(563.41,441.32)(0.509,-0.740){19}{\rule{0.123pt}{1.369pt}}
\multiput(560.34,444.16)(13.000,-16.158){2}{\rule{0.800pt}{0.685pt}}
\multiput(576.41,428.00)(0.509,0.740){19}{\rule{0.123pt}{1.369pt}}
\multiput(573.34,428.00)(13.000,16.158){2}{\rule{0.800pt}{0.685pt}}
\multiput(589.41,441.66)(0.509,-0.683){21}{\rule{0.123pt}{1.286pt}}
\multiput(586.34,444.33)(14.000,-16.331){2}{\rule{0.800pt}{0.643pt}}
\put(509.0,447.0){\rule[-0.400pt]{12.768pt}{0.800pt}}
\multiput(616.41,428.00)(0.509,0.740){19}{\rule{0.123pt}{1.369pt}}
\multiput(613.34,428.00)(13.000,16.158){2}{\rule{0.800pt}{0.685pt}}
\multiput(629.41,441.66)(0.509,-0.683){21}{\rule{0.123pt}{1.286pt}}
\multiput(626.34,444.33)(14.000,-16.331){2}{\rule{0.800pt}{0.643pt}}
\multiput(643.41,428.00)(0.509,0.740){19}{\rule{0.123pt}{1.369pt}}
\multiput(640.34,428.00)(13.000,16.158){2}{\rule{0.800pt}{0.685pt}}
\multiput(656.41,441.32)(0.509,-0.740){19}{\rule{0.123pt}{1.369pt}}
\multiput(653.34,444.16)(13.000,-16.158){2}{\rule{0.800pt}{0.685pt}}
\put(602.0,428.0){\rule[-0.400pt]{3.132pt}{0.800pt}}
\multiput(682.41,428.00)(0.509,1.408){21}{\rule{0.123pt}{2.371pt}}
\multiput(679.34,428.00)(14.000,33.078){2}{\rule{0.800pt}{1.186pt}}
\multiput(695.00,467.41)(0.492,0.509){19}{\rule{1.000pt}{0.123pt}}
\multiput(695.00,464.34)(10.924,13.000){2}{\rule{0.500pt}{0.800pt}}
\put(668.0,428.0){\rule[-0.400pt]{3.132pt}{0.800pt}}
\multiput(722.41,479.00)(0.509,9.303){21}{\rule{0.123pt}{14.200pt}}
\multiput(719.34,479.00)(14.000,215.527){2}{\rule{0.800pt}{7.100pt}}
\put(708.0,479.0){\rule[-0.400pt]{3.132pt}{0.800pt}}
\multiput(802.41,665.69)(0.509,-9.261){19}{\rule{0.123pt}{14.046pt}}
\multiput(799.34,694.85)(13.000,-195.847){2}{\rule{0.800pt}{7.023pt}}
\multiput(815.41,499.00)(0.509,8.540){21}{\rule{0.123pt}{13.057pt}}
\multiput(812.34,499.00)(14.000,197.899){2}{\rule{0.800pt}{6.529pt}}
\put(735.0,724.0){\rule[-0.400pt]{15.899pt}{0.800pt}}
\multiput(855.41,698.74)(0.509,-3.887){21}{\rule{0.123pt}{6.086pt}}
\multiput(852.34,711.37)(14.000,-90.369){2}{\rule{0.800pt}{3.043pt}}
\multiput(869.41,621.00)(0.509,4.214){19}{\rule{0.123pt}{6.538pt}}
\multiput(866.34,621.00)(13.000,89.429){2}{\rule{0.800pt}{3.269pt}}
\put(828.0,724.0){\rule[-0.400pt]{6.263pt}{0.800pt}}
\multiput(895.41,715.00)(0.509,-1.278){19}{\rule{0.123pt}{2.169pt}}
\multiput(892.34,719.50)(13.000,-27.498){2}{\rule{0.800pt}{1.085pt}}
\multiput(908.41,692.00)(0.509,1.179){21}{\rule{0.123pt}{2.029pt}}
\multiput(905.34,692.00)(14.000,27.790){2}{\rule{0.800pt}{1.014pt}}
\put(881.0,724.0){\rule[-0.400pt]{3.132pt}{0.800pt}}
\multiput(948.41,715.58)(0.509,-1.179){21}{\rule{0.123pt}{2.029pt}}
\multiput(945.34,719.79)(14.000,-27.790){2}{\rule{0.800pt}{1.014pt}}
\multiput(962.41,673.03)(0.509,-2.891){19}{\rule{0.123pt}{4.569pt}}
\multiput(959.34,682.52)(13.000,-61.516){2}{\rule{0.800pt}{2.285pt}}
\multiput(975.41,621.00)(0.509,2.105){19}{\rule{0.123pt}{3.400pt}}
\multiput(972.34,621.00)(13.000,44.943){2}{\rule{0.800pt}{1.700pt}}
\multiput(988.41,658.89)(0.509,-2.105){19}{\rule{0.123pt}{3.400pt}}
\multiput(985.34,665.94)(13.000,-44.943){2}{\rule{0.800pt}{1.700pt}}
\put(921.0,724.0){\rule[-0.400pt]{6.263pt}{0.800pt}}
\multiput(1134.41,621.00)(0.509,1.942){21}{\rule{0.123pt}{3.171pt}}
\multiput(1131.34,621.00)(14.000,45.418){2}{\rule{0.800pt}{1.586pt}}
\multiput(1148.41,664.00)(0.509,-1.278){19}{\rule{0.123pt}{2.169pt}}
\multiput(1145.34,668.50)(13.000,-27.498){2}{\rule{0.800pt}{1.085pt}}
\put(1000.0,621.0){\rule[-0.400pt]{32.040pt}{0.800pt}}
\multiput(1187.41,635.43)(0.509,-0.721){21}{\rule{0.123pt}{1.343pt}}
\multiput(1184.34,638.21)(14.000,-17.213){2}{\rule{0.800pt}{0.671pt}}
\put(1160.0,641.0){\rule[-0.400pt]{6.263pt}{0.800pt}}
\multiput(1254.41,621.00)(0.509,0.781){19}{\rule{0.123pt}{1.431pt}}
\multiput(1251.34,621.00)(13.000,17.030){2}{\rule{0.800pt}{0.715pt}}
\multiput(1267.41,635.43)(0.509,-0.721){21}{\rule{0.123pt}{1.343pt}}
\multiput(1264.34,638.21)(14.000,-17.213){2}{\rule{0.800pt}{0.671pt}}
\put(1200.0,621.0){\rule[-0.400pt]{12.768pt}{0.800pt}}
\put(1280.0,621.0){\rule[-0.400pt]{38.303pt}{0.800pt}}
\sbox{\plotpoint}{\rule[-0.200pt]{0.400pt}{0.400pt}}%
\put(1279,695){\makebox(0,0)[r]{Castigos}}
\put(1299.0,695.0){\rule[-0.200pt]{24.090pt}{0.400pt}}
\put(123,131){\usebox{\plotpoint}}
\put(123.0,131.0){\rule[-0.200pt]{317.024pt}{0.400pt}}
\put(1279,654){\makebox(0,0)[r]{Max. carga}}
\multiput(1299,654)(20.756,0.000){5}{\usebox{\plotpoint}}
\put(1399,654){\usebox{\plotpoint}}
\put(123,621){\usebox{\plotpoint}}
\multiput(123,621)(9.840,18.275){2}{\usebox{\plotpoint}}
\put(148.98,647.00){\usebox{\plotpoint}}
\put(169.74,647.00){\usebox{\plotpoint}}
\put(182.87,634.24){\usebox{\plotpoint}}
\put(195.72,621.00){\usebox{\plotpoint}}
\put(216.47,621.00){\usebox{\plotpoint}}
\put(237.23,621.00){\usebox{\plotpoint}}
\multiput(243,621)(3.412,20.473){4}{\usebox{\plotpoint}}
\multiput(256,699)(3.412,-20.473){3}{\usebox{\plotpoint}}
\put(271.12,621.00){\usebox{\plotpoint}}
\put(291.88,621.00){\usebox{\plotpoint}}
\put(303.44,635.88){\usebox{\plotpoint}}
\multiput(309,647)(9.840,-18.275){2}{\usebox{\plotpoint}}
\put(343.30,621.00){\usebox{\plotpoint}}
\multiput(363,621)(9.282,18.564){2}{\usebox{\plotpoint}}
\put(389.50,647.00){\usebox{\plotpoint}}
\put(410.25,647.00){\usebox{\plotpoint}}
\put(431.01,647.00){\usebox{\plotpoint}}
\put(451.76,647.00){\usebox{\plotpoint}}
\put(472.52,647.00){\usebox{\plotpoint}}
\put(493.28,647.00){\usebox{\plotpoint}}
\put(514.03,647.00){\usebox{\plotpoint}}
\put(534.79,647.00){\usebox{\plotpoint}}
\put(555.54,647.00){\usebox{\plotpoint}}
\put(576.30,647.00){\usebox{\plotpoint}}
\put(597.05,647.00){\usebox{\plotpoint}}
\put(617.81,647.00){\usebox{\plotpoint}}
\put(638.56,647.00){\usebox{\plotpoint}}
\put(659.32,647.00){\usebox{\plotpoint}}
\put(680.07,647.00){\usebox{\plotpoint}}
\put(700.83,647.00){\usebox{\plotpoint}}
\multiput(721,647)(9.840,18.275){2}{\usebox{\plotpoint}}
\put(747.57,673.00){\usebox{\plotpoint}}
\put(768.32,673.00){\usebox{\plotpoint}}
\put(789.08,673.00){\usebox{\plotpoint}}
\multiput(801,673)(5.034,-20.136){3}{\usebox{\plotpoint}}
\multiput(814,621)(5.396,20.042){2}{\usebox{\plotpoint}}
\put(833.16,673.00){\usebox{\plotpoint}}
\put(853.91,673.00){\usebox{\plotpoint}}
\put(863.80,654.80){\usebox{\plotpoint}}
\put(873.32,657.64){\usebox{\plotpoint}}
\put(884.58,673.00){\usebox{\plotpoint}}
\multiput(894,673)(5.034,-20.136){3}{\usebox{\plotpoint}}
\multiput(907,621)(5.396,20.042){2}{\usebox{\plotpoint}}
\put(928.66,673.00){\usebox{\plotpoint}}
\multiput(947,673)(5.396,-20.042){3}{\usebox{\plotpoint}}
\put(971.83,621.00){\usebox{\plotpoint}}
\put(982.31,637.63){\usebox{\plotpoint}}
\put(991.60,637.81){\usebox{\plotpoint}}
\put(1001.96,621.00){\usebox{\plotpoint}}
\put(1022.72,621.00){\usebox{\plotpoint}}
\put(1043.47,621.00){\usebox{\plotpoint}}
\put(1064.23,621.00){\usebox{\plotpoint}}
\put(1084.98,621.00){\usebox{\plotpoint}}
\put(1105.74,621.00){\usebox{\plotpoint}}
\put(1126.50,621.00){\usebox{\plotpoint}}
\put(1139.76,633.55){\usebox{\plotpoint}}
\multiput(1147,647)(9.282,-18.564){2}{\usebox{\plotpoint}}
\put(1177.92,621.00){\usebox{\plotpoint}}
\put(1198.67,621.00){\usebox{\plotpoint}}
\put(1219.43,621.00){\usebox{\plotpoint}}
\put(1240.19,621.00){\usebox{\plotpoint}}
\put(1260.94,621.00){\usebox{\plotpoint}}
\put(1281.70,621.00){\usebox{\plotpoint}}
\put(1302.45,621.00){\usebox{\plotpoint}}
\put(1323.21,621.00){\usebox{\plotpoint}}
\put(1343.96,621.00){\usebox{\plotpoint}}
\put(1364.72,621.00){\usebox{\plotpoint}}
\put(1385.47,621.00){\usebox{\plotpoint}}
\put(1406.23,621.00){\usebox{\plotpoint}}
\put(1426.98,621.00){\usebox{\plotpoint}}
\put(1439,621){\usebox{\plotpoint}}
\put(110.0,131.0){\rule[-0.200pt]{0.400pt}{155.380pt}}
\put(110.0,131.0){\rule[-0.200pt]{320.156pt}{0.400pt}}
\put(1439.0,131.0){\rule[-0.200pt]{0.400pt}{155.380pt}}
\put(110.0,776.0){\rule[-0.200pt]{320.156pt}{0.400pt}}
\end{picture}

\end{center}


\subsection{Importancia Elitismo}
Como se aprecio en los experimentos antes mecionados, y debido al alto grado de
conectividad entre los ramos y las restricciones, los operadores tienden a
empeorar la calidad de las poblaciones de forma natural, por lo que el uso de
elitismo es critico en la obtención de un resultado óptimo. El gráfico 3 muestra
como la inclusión de elitismos mejora inmediatamente el rendimiento de los
operadores simples (comparar con gráfico 1).


\begin{center}
  % GNUPLOT: LaTeX picture
\setlength{\unitlength}{0.240900pt}
\ifx\plotpoint\undefined\newsavebox{\plotpoint}\fi
\sbox{\plotpoint}{\rule[-0.200pt]{0.400pt}{0.400pt}}%
\begin{picture}(1500,900)(0,0)
\sbox{\plotpoint}{\rule[-0.200pt]{0.400pt}{0.400pt}}%
\put(110.0,131.0){\rule[-0.200pt]{4.818pt}{0.400pt}}
\put(90,131){\makebox(0,0)[r]{ 0}}
\put(1419.0,131.0){\rule[-0.200pt]{4.818pt}{0.400pt}}
\put(110.0,196.0){\rule[-0.200pt]{4.818pt}{0.400pt}}
\put(90,196){\makebox(0,0)[r]{ 2}}
\put(1419.0,196.0){\rule[-0.200pt]{4.818pt}{0.400pt}}
\put(110.0,260.0){\rule[-0.200pt]{4.818pt}{0.400pt}}
\put(90,260){\makebox(0,0)[r]{ 4}}
\put(1419.0,260.0){\rule[-0.200pt]{4.818pt}{0.400pt}}
\put(110.0,325.0){\rule[-0.200pt]{4.818pt}{0.400pt}}
\put(90,325){\makebox(0,0)[r]{ 6}}
\put(1419.0,325.0){\rule[-0.200pt]{4.818pt}{0.400pt}}
\put(110.0,389.0){\rule[-0.200pt]{4.818pt}{0.400pt}}
\put(90,389){\makebox(0,0)[r]{ 8}}
\put(1419.0,389.0){\rule[-0.200pt]{4.818pt}{0.400pt}}
\put(110.0,454.0){\rule[-0.200pt]{4.818pt}{0.400pt}}
\put(90,454){\makebox(0,0)[r]{ 10}}
\put(1419.0,454.0){\rule[-0.200pt]{4.818pt}{0.400pt}}
\put(110.0,518.0){\rule[-0.200pt]{4.818pt}{0.400pt}}
\put(90,518){\makebox(0,0)[r]{ 12}}
\put(1419.0,518.0){\rule[-0.200pt]{4.818pt}{0.400pt}}
\put(110.0,583.0){\rule[-0.200pt]{4.818pt}{0.400pt}}
\put(90,583){\makebox(0,0)[r]{ 14}}
\put(1419.0,583.0){\rule[-0.200pt]{4.818pt}{0.400pt}}
\put(110.0,647.0){\rule[-0.200pt]{4.818pt}{0.400pt}}
\put(90,647){\makebox(0,0)[r]{ 16}}
\put(1419.0,647.0){\rule[-0.200pt]{4.818pt}{0.400pt}}
\put(110.0,712.0){\rule[-0.200pt]{4.818pt}{0.400pt}}
\put(90,712){\makebox(0,0)[r]{ 18}}
\put(1419.0,712.0){\rule[-0.200pt]{4.818pt}{0.400pt}}
\put(110.0,776.0){\rule[-0.200pt]{4.818pt}{0.400pt}}
\put(90,776){\makebox(0,0)[r]{ 20}}
\put(1419.0,776.0){\rule[-0.200pt]{4.818pt}{0.400pt}}
\put(110.0,131.0){\rule[-0.200pt]{0.400pt}{4.818pt}}
\put(110,90){\makebox(0,0){ 0}}
\put(110.0,756.0){\rule[-0.200pt]{0.400pt}{4.818pt}}
\put(243.0,131.0){\rule[-0.200pt]{0.400pt}{4.818pt}}
\put(243,90){\makebox(0,0){ 10}}
\put(243.0,756.0){\rule[-0.200pt]{0.400pt}{4.818pt}}
\put(376.0,131.0){\rule[-0.200pt]{0.400pt}{4.818pt}}
\put(376,90){\makebox(0,0){ 20}}
\put(376.0,756.0){\rule[-0.200pt]{0.400pt}{4.818pt}}
\put(509.0,131.0){\rule[-0.200pt]{0.400pt}{4.818pt}}
\put(509,90){\makebox(0,0){ 30}}
\put(509.0,756.0){\rule[-0.200pt]{0.400pt}{4.818pt}}
\put(642.0,131.0){\rule[-0.200pt]{0.400pt}{4.818pt}}
\put(642,90){\makebox(0,0){ 40}}
\put(642.0,756.0){\rule[-0.200pt]{0.400pt}{4.818pt}}
\put(774.0,131.0){\rule[-0.200pt]{0.400pt}{4.818pt}}
\put(774,90){\makebox(0,0){ 50}}
\put(774.0,756.0){\rule[-0.200pt]{0.400pt}{4.818pt}}
\put(907.0,131.0){\rule[-0.200pt]{0.400pt}{4.818pt}}
\put(907,90){\makebox(0,0){ 60}}
\put(907.0,756.0){\rule[-0.200pt]{0.400pt}{4.818pt}}
\put(1040.0,131.0){\rule[-0.200pt]{0.400pt}{4.818pt}}
\put(1040,90){\makebox(0,0){ 70}}
\put(1040.0,756.0){\rule[-0.200pt]{0.400pt}{4.818pt}}
\put(1173.0,131.0){\rule[-0.200pt]{0.400pt}{4.818pt}}
\put(1173,90){\makebox(0,0){ 80}}
\put(1173.0,756.0){\rule[-0.200pt]{0.400pt}{4.818pt}}
\put(1306.0,131.0){\rule[-0.200pt]{0.400pt}{4.818pt}}
\put(1306,90){\makebox(0,0){ 90}}
\put(1306.0,756.0){\rule[-0.200pt]{0.400pt}{4.818pt}}
\put(1439.0,131.0){\rule[-0.200pt]{0.400pt}{4.818pt}}
\put(1439,90){\makebox(0,0){ 100}}
\put(1439.0,756.0){\rule[-0.200pt]{0.400pt}{4.818pt}}
\put(110.0,131.0){\rule[-0.200pt]{0.400pt}{155.380pt}}
\put(110.0,131.0){\rule[-0.200pt]{320.156pt}{0.400pt}}
\put(1439.0,131.0){\rule[-0.200pt]{0.400pt}{155.380pt}}
\put(110.0,776.0){\rule[-0.200pt]{320.156pt}{0.400pt}}
\put(774,29){\makebox(0,0){Generaciones}}
\put(774,838){\makebox(0,0){BACP 8 - Mut. y Cruz. Simple con elitismo}}
\sbox{\plotpoint}{\rule[-0.400pt]{0.800pt}{0.800pt}}%
\sbox{\plotpoint}{\rule[-0.200pt]{0.400pt}{0.400pt}}%
\put(1279,736){\makebox(0,0)[r]{Mejor}}
\sbox{\plotpoint}{\rule[-0.400pt]{0.800pt}{0.800pt}}%
\put(1299.0,736.0){\rule[-0.400pt]{24.090pt}{0.800pt}}
\put(123,534){\usebox{\plotpoint}}
\multiput(377.41,527.04)(0.509,-0.947){19}{\rule{0.123pt}{1.677pt}}
\multiput(374.34,530.52)(13.000,-20.519){2}{\rule{0.800pt}{0.838pt}}
\put(123.0,534.0){\rule[-0.400pt]{60.948pt}{0.800pt}}
\multiput(1121.41,492.57)(0.509,-2.643){19}{\rule{0.123pt}{4.200pt}}
\multiput(1118.34,501.28)(13.000,-56.283){2}{\rule{0.800pt}{2.100pt}}
\put(389.0,510.0){\rule[-0.400pt]{176.098pt}{0.800pt}}
\multiput(1307.41,411.22)(0.509,-5.290){19}{\rule{0.123pt}{8.138pt}}
\multiput(1304.34,428.11)(13.000,-112.108){2}{\rule{0.800pt}{4.069pt}}
\put(1133.0,445.0){\rule[-0.400pt]{41.676pt}{0.800pt}}
\put(1319.0,316.0){\rule[-0.400pt]{28.908pt}{0.800pt}}
\sbox{\plotpoint}{\rule[-0.200pt]{0.400pt}{0.400pt}}%
\put(1279,695){\makebox(0,0)[r]{Castigos}}
\put(1299.0,695.0){\rule[-0.200pt]{24.090pt}{0.400pt}}
\put(123,131){\usebox{\plotpoint}}
\put(123.0,131.0){\rule[-0.200pt]{317.024pt}{0.400pt}}
\put(1279,654){\makebox(0,0)[r]{Max. carga}}
\multiput(1299,654)(20.756,0.000){5}{\usebox{\plotpoint}}
\put(1399,654){\usebox{\plotpoint}}
\put(123,744){\usebox{\plotpoint}}
\put(123.00,744.00){\usebox{\plotpoint}}
\put(143.76,744.00){\usebox{\plotpoint}}
\put(164.51,744.00){\usebox{\plotpoint}}
\put(185.27,744.00){\usebox{\plotpoint}}
\put(206.02,744.00){\usebox{\plotpoint}}
\put(226.78,744.00){\usebox{\plotpoint}}
\put(247.53,744.00){\usebox{\plotpoint}}
\put(268.29,744.00){\usebox{\plotpoint}}
\put(289.04,744.00){\usebox{\plotpoint}}
\put(309.80,744.00){\usebox{\plotpoint}}
\put(330.56,744.00){\usebox{\plotpoint}}
\put(351.31,744.00){\usebox{\plotpoint}}
\put(372.07,744.00){\usebox{\plotpoint}}
\put(392.82,744.00){\usebox{\plotpoint}}
\put(413.58,744.00){\usebox{\plotpoint}}
\put(434.33,744.00){\usebox{\plotpoint}}
\put(455.09,744.00){\usebox{\plotpoint}}
\put(475.84,744.00){\usebox{\plotpoint}}
\put(496.60,744.00){\usebox{\plotpoint}}
\put(517.35,744.00){\usebox{\plotpoint}}
\put(538.11,744.00){\usebox{\plotpoint}}
\put(558.87,744.00){\usebox{\plotpoint}}
\put(579.62,744.00){\usebox{\plotpoint}}
\put(600.38,744.00){\usebox{\plotpoint}}
\put(621.13,744.00){\usebox{\plotpoint}}
\put(641.89,744.00){\usebox{\plotpoint}}
\put(662.64,744.00){\usebox{\plotpoint}}
\put(683.40,744.00){\usebox{\plotpoint}}
\put(704.15,744.00){\usebox{\plotpoint}}
\put(724.91,744.00){\usebox{\plotpoint}}
\put(745.66,744.00){\usebox{\plotpoint}}
\put(766.42,744.00){\usebox{\plotpoint}}
\put(787.18,744.00){\usebox{\plotpoint}}
\put(807.93,744.00){\usebox{\plotpoint}}
\put(828.69,744.00){\usebox{\plotpoint}}
\put(849.44,744.00){\usebox{\plotpoint}}
\put(870.20,744.00){\usebox{\plotpoint}}
\put(890.95,744.00){\usebox{\plotpoint}}
\put(911.71,744.00){\usebox{\plotpoint}}
\put(932.46,744.00){\usebox{\plotpoint}}
\put(953.22,744.00){\usebox{\plotpoint}}
\put(973.98,744.00){\usebox{\plotpoint}}
\put(994.73,744.00){\usebox{\plotpoint}}
\put(1015.49,744.00){\usebox{\plotpoint}}
\put(1036.24,744.00){\usebox{\plotpoint}}
\put(1057.00,744.00){\usebox{\plotpoint}}
\put(1077.75,744.00){\usebox{\plotpoint}}
\put(1098.51,744.00){\usebox{\plotpoint}}
\put(1119.26,744.00){\usebox{\plotpoint}}
\put(1140.02,744.00){\usebox{\plotpoint}}
\put(1160.77,744.00){\usebox{\plotpoint}}
\put(1181.53,744.00){\usebox{\plotpoint}}
\put(1202.29,744.00){\usebox{\plotpoint}}
\put(1223.04,744.00){\usebox{\plotpoint}}
\put(1243.80,744.00){\usebox{\plotpoint}}
\put(1264.55,744.00){\usebox{\plotpoint}}
\put(1285.31,744.00){\usebox{\plotpoint}}
\multiput(1306,744)(7.812,-19.229){2}{\usebox{\plotpoint}}
\put(1326.03,712.00){\usebox{\plotpoint}}
\put(1346.79,712.00){\usebox{\plotpoint}}
\put(1367.55,712.00){\usebox{\plotpoint}}
\put(1388.30,712.00){\usebox{\plotpoint}}
\put(1409.06,712.00){\usebox{\plotpoint}}
\put(1429.81,712.00){\usebox{\plotpoint}}
\put(1439,712){\usebox{\plotpoint}}
\put(110.0,131.0){\rule[-0.200pt]{0.400pt}{155.380pt}}
\put(110.0,131.0){\rule[-0.200pt]{320.156pt}{0.400pt}}
\put(1439.0,131.0){\rule[-0.200pt]{0.400pt}{155.380pt}}
\put(110.0,776.0){\rule[-0.200pt]{320.156pt}{0.400pt}}
\end{picture}

\end{center}


\subsection{Otros experimentos menos importantes}
\subsubsection{Restart en el elitismo}
Buscando lidear con la convergencia prematura, se implemento un mecanismo de
restart para el elitismo acompañado de una explosión en la  mutación luego de un cierto
número de generaciones estancadas. Estas estrategias no lograron dar resultados
convincentes ya que nuevamente las generaciones sin elitismo solo tendían a
alejarse del óptimo encontrado hasta ese momento.
\subsubsection{Tiempo de ejecución y Restarts}
El tiempo de ejecución, en todo los casos, es despreciable. Lo
anterior junto a la calidad de las soluciones en ejecuciones
aleatorias hizo que el número de restart no fuera considerado un
parámetro critico quedando determinado a 10 o 20 de forme arbitraria.



\section{Resultados}
Utilizando los valores y las estrategias desprendidas de la sección anterior, se
presentan los mejores resultados de ejecuciones casuales no superiores a 10
restarts para caso donde:

\begin{itemize}
  \item \textbf{Mejor:} Corresponde valor del mejor individuo por generación.
  \item \textbf{Castigo:} El número de restricciones rotas.
  \item \textbf{Max. Carga:} Corresponde a la carga máxima de créditos observada en algún semestre.
\end{itemize}


\begin{center}
% GNUPLOT: LaTeX picture
\setlength{\unitlength}{0.240900pt}
\ifx\plotpoint\undefined\newsavebox{\plotpoint}\fi
\sbox{\plotpoint}{\rule[-0.200pt]{0.400pt}{0.400pt}}%
\begin{picture}(1500,900)(0,0)
\sbox{\plotpoint}{\rule[-0.200pt]{0.400pt}{0.400pt}}%
\put(110.0,131.0){\rule[-0.200pt]{4.818pt}{0.400pt}}
\put(90,131){\makebox(0,0)[r]{ 0}}
\put(1419.0,131.0){\rule[-0.200pt]{4.818pt}{0.400pt}}
\put(110.0,292.0){\rule[-0.200pt]{4.818pt}{0.400pt}}
\put(90,292){\makebox(0,0)[r]{ 5}}
\put(1419.0,292.0){\rule[-0.200pt]{4.818pt}{0.400pt}}
\put(110.0,454.0){\rule[-0.200pt]{4.818pt}{0.400pt}}
\put(90,454){\makebox(0,0)[r]{ 10}}
\put(1419.0,454.0){\rule[-0.200pt]{4.818pt}{0.400pt}}
\put(110.0,615.0){\rule[-0.200pt]{4.818pt}{0.400pt}}
\put(90,615){\makebox(0,0)[r]{ 15}}
\put(1419.0,615.0){\rule[-0.200pt]{4.818pt}{0.400pt}}
\put(110.0,776.0){\rule[-0.200pt]{4.818pt}{0.400pt}}
\put(90,776){\makebox(0,0)[r]{ 20}}
\put(1419.0,776.0){\rule[-0.200pt]{4.818pt}{0.400pt}}
\put(110.0,131.0){\rule[-0.200pt]{0.400pt}{4.818pt}}
\put(110,90){\makebox(0,0){ 0}}
\put(110.0,756.0){\rule[-0.200pt]{0.400pt}{4.818pt}}
\put(243.0,131.0){\rule[-0.200pt]{0.400pt}{4.818pt}}
\put(243,90){\makebox(0,0){ 10}}
\put(243.0,756.0){\rule[-0.200pt]{0.400pt}{4.818pt}}
\put(376.0,131.0){\rule[-0.200pt]{0.400pt}{4.818pt}}
\put(376,90){\makebox(0,0){ 20}}
\put(376.0,756.0){\rule[-0.200pt]{0.400pt}{4.818pt}}
\put(509.0,131.0){\rule[-0.200pt]{0.400pt}{4.818pt}}
\put(509,90){\makebox(0,0){ 30}}
\put(509.0,756.0){\rule[-0.200pt]{0.400pt}{4.818pt}}
\put(642.0,131.0){\rule[-0.200pt]{0.400pt}{4.818pt}}
\put(642,90){\makebox(0,0){ 40}}
\put(642.0,756.0){\rule[-0.200pt]{0.400pt}{4.818pt}}
\put(774.0,131.0){\rule[-0.200pt]{0.400pt}{4.818pt}}
\put(774,90){\makebox(0,0){ 50}}
\put(774.0,756.0){\rule[-0.200pt]{0.400pt}{4.818pt}}
\put(907.0,131.0){\rule[-0.200pt]{0.400pt}{4.818pt}}
\put(907,90){\makebox(0,0){ 60}}
\put(907.0,756.0){\rule[-0.200pt]{0.400pt}{4.818pt}}
\put(1040.0,131.0){\rule[-0.200pt]{0.400pt}{4.818pt}}
\put(1040,90){\makebox(0,0){ 70}}
\put(1040.0,756.0){\rule[-0.200pt]{0.400pt}{4.818pt}}
\put(1173.0,131.0){\rule[-0.200pt]{0.400pt}{4.818pt}}
\put(1173,90){\makebox(0,0){ 80}}
\put(1173.0,756.0){\rule[-0.200pt]{0.400pt}{4.818pt}}
\put(1306.0,131.0){\rule[-0.200pt]{0.400pt}{4.818pt}}
\put(1306,90){\makebox(0,0){ 90}}
\put(1306.0,756.0){\rule[-0.200pt]{0.400pt}{4.818pt}}
\put(1439.0,131.0){\rule[-0.200pt]{0.400pt}{4.818pt}}
\put(1439,90){\makebox(0,0){ 100}}
\put(1439.0,756.0){\rule[-0.200pt]{0.400pt}{4.818pt}}
\put(110.0,131.0){\rule[-0.200pt]{0.400pt}{155.380pt}}
\put(110.0,131.0){\rule[-0.200pt]{320.156pt}{0.400pt}}
\put(1439.0,131.0){\rule[-0.200pt]{0.400pt}{155.380pt}}
\put(110.0,776.0){\rule[-0.200pt]{320.156pt}{0.400pt}}
\put(774,29){\makebox(0,0){Generaciones}}
\put(774,838){\makebox(0,0){Gráfico 4: BACP 8 - Mejor resultado}}
\sbox{\plotpoint}{\rule[-0.400pt]{0.800pt}{0.800pt}}%
\sbox{\plotpoint}{\rule[-0.200pt]{0.400pt}{0.400pt}}%
\put(1279,736){\makebox(0,0)[r]{Mejor}}
\sbox{\plotpoint}{\rule[-0.400pt]{0.800pt}{0.800pt}}%
\put(1299.0,736.0){\rule[-0.400pt]{24.090pt}{0.800pt}}
\put(123,550){\usebox{\plotpoint}}
\multiput(124.41,539.68)(0.509,-1.484){21}{\rule{0.123pt}{2.486pt}}
\multiput(121.34,544.84)(14.000,-34.841){2}{\rule{0.800pt}{1.243pt}}
\multiput(164.41,480.30)(0.509,-4.628){19}{\rule{0.123pt}{7.154pt}}
\multiput(161.34,495.15)(13.000,-98.152){2}{\rule{0.800pt}{3.577pt}}
\put(137.0,510.0){\rule[-0.400pt]{6.263pt}{0.800pt}}
\multiput(217.41,386.68)(0.509,-1.484){21}{\rule{0.123pt}{2.486pt}}
\multiput(214.34,391.84)(14.000,-34.841){2}{\rule{0.800pt}{1.243pt}}
\put(176.0,397.0){\rule[-0.400pt]{9.636pt}{0.800pt}}
\multiput(457.41,345.70)(0.509,-1.650){19}{\rule{0.123pt}{2.723pt}}
\multiput(454.34,351.35)(13.000,-35.348){2}{\rule{0.800pt}{1.362pt}}
\put(230.0,357.0){\rule[-0.400pt]{54.443pt}{0.800pt}}
\multiput(789.41,298.82)(0.509,-2.601){19}{\rule{0.123pt}{4.138pt}}
\multiput(786.34,307.41)(13.000,-55.410){2}{\rule{0.800pt}{2.069pt}}
\put(469.0,316.0){\rule[-0.400pt]{76.847pt}{0.800pt}}
\put(801.0,252.0){\rule[-0.400pt]{153.694pt}{0.800pt}}
\sbox{\plotpoint}{\rule[-0.200pt]{0.400pt}{0.400pt}}%
\put(1279,695){\makebox(0,0)[r]{Castigos}}
\put(1299.0,695.0){\rule[-0.200pt]{24.090pt}{0.400pt}}
\put(123,131){\usebox{\plotpoint}}
\put(123.0,131.0){\rule[-0.200pt]{317.024pt}{0.400pt}}
\put(1279,654){\makebox(0,0)[r]{Max. carga}}
\multiput(1299,654)(20.756,0.000){5}{\usebox{\plotpoint}}
\put(1399,654){\usebox{\plotpoint}}
\put(123,776){\usebox{\plotpoint}}
\multiput(123,776)(8.319,-19.015){2}{\usebox{\plotpoint}}
\put(143.58,744.00){\usebox{\plotpoint}}
\multiput(163,744)(7.812,19.229){2}{\usebox{\plotpoint}}
\put(184.31,776.00){\usebox{\plotpoint}}
\put(205.06,776.00){\usebox{\plotpoint}}
\multiput(216,776)(8.319,-19.015){2}{\usebox{\plotpoint}}
\put(246.40,744.00){\usebox{\plotpoint}}
\put(267.16,744.00){\usebox{\plotpoint}}
\put(287.91,744.00){\usebox{\plotpoint}}
\put(308.67,744.00){\usebox{\plotpoint}}
\put(329.42,744.00){\usebox{\plotpoint}}
\put(350.18,744.00){\usebox{\plotpoint}}
\put(370.94,744.00){\usebox{\plotpoint}}
\put(391.69,744.00){\usebox{\plotpoint}}
\put(412.45,744.00){\usebox{\plotpoint}}
\put(433.20,744.00){\usebox{\plotpoint}}
\put(453.96,744.00){\usebox{\plotpoint}}
\put(463.04,726.66){\usebox{\plotpoint}}
\put(473.93,712.00){\usebox{\plotpoint}}
\put(494.68,712.00){\usebox{\plotpoint}}
\put(515.44,712.00){\usebox{\plotpoint}}
\put(536.20,712.00){\usebox{\plotpoint}}
\put(556.95,712.00){\usebox{\plotpoint}}
\put(577.71,712.00){\usebox{\plotpoint}}
\put(598.46,712.00){\usebox{\plotpoint}}
\put(619.22,712.00){\usebox{\plotpoint}}
\put(639.97,712.00){\usebox{\plotpoint}}
\put(660.73,712.00){\usebox{\plotpoint}}
\put(681.48,712.00){\usebox{\plotpoint}}
\put(702.24,712.00){\usebox{\plotpoint}}
\put(722.99,712.00){\usebox{\plotpoint}}
\put(743.75,712.00){\usebox{\plotpoint}}
\put(764.51,712.00){\usebox{\plotpoint}}
\put(785.26,712.00){\usebox{\plotpoint}}
\put(794.60,695.24){\usebox{\plotpoint}}
\put(804.30,679.00){\usebox{\plotpoint}}
\put(825.06,679.00){\usebox{\plotpoint}}
\put(845.81,679.00){\usebox{\plotpoint}}
\put(866.57,679.00){\usebox{\plotpoint}}
\put(887.33,679.00){\usebox{\plotpoint}}
\put(908.08,679.00){\usebox{\plotpoint}}
\put(928.84,679.00){\usebox{\plotpoint}}
\put(949.59,679.00){\usebox{\plotpoint}}
\put(970.35,679.00){\usebox{\plotpoint}}
\put(991.10,679.00){\usebox{\plotpoint}}
\put(1011.86,679.00){\usebox{\plotpoint}}
\put(1032.61,679.00){\usebox{\plotpoint}}
\put(1053.37,679.00){\usebox{\plotpoint}}
\put(1074.13,679.00){\usebox{\plotpoint}}
\put(1094.88,679.00){\usebox{\plotpoint}}
\put(1115.64,679.00){\usebox{\plotpoint}}
\put(1136.39,679.00){\usebox{\plotpoint}}
\put(1157.15,679.00){\usebox{\plotpoint}}
\put(1177.90,679.00){\usebox{\plotpoint}}
\put(1198.66,679.00){\usebox{\plotpoint}}
\put(1219.41,679.00){\usebox{\plotpoint}}
\put(1240.17,679.00){\usebox{\plotpoint}}
\put(1260.92,679.00){\usebox{\plotpoint}}
\put(1281.68,679.00){\usebox{\plotpoint}}
\put(1302.44,679.00){\usebox{\plotpoint}}
\put(1323.19,679.00){\usebox{\plotpoint}}
\put(1343.95,679.00){\usebox{\plotpoint}}
\put(1364.70,679.00){\usebox{\plotpoint}}
\put(1385.46,679.00){\usebox{\plotpoint}}
\put(1406.21,679.00){\usebox{\plotpoint}}
\put(1426.97,679.00){\usebox{\plotpoint}}
\put(1439,679){\usebox{\plotpoint}}
\put(110.0,131.0){\rule[-0.200pt]{0.400pt}{155.380pt}}
\put(110.0,131.0){\rule[-0.200pt]{320.156pt}{0.400pt}}
\put(1439.0,131.0){\rule[-0.200pt]{0.400pt}{155.380pt}}
\put(110.0,776.0){\rule[-0.200pt]{320.156pt}{0.400pt}}
\end{picture}

\end{center}
\begin{center}
% GNUPLOT: LaTeX picture
\setlength{\unitlength}{0.240900pt}
\ifx\plotpoint\undefined\newsavebox{\plotpoint}\fi
\sbox{\plotpoint}{\rule[-0.200pt]{0.400pt}{0.400pt}}%
\begin{picture}(1500,900)(0,0)
\sbox{\plotpoint}{\rule[-0.200pt]{0.400pt}{0.400pt}}%
\put(110.0,131.0){\rule[-0.200pt]{4.818pt}{0.400pt}}
\put(90,131){\makebox(0,0)[r]{ 0}}
\put(1419.0,131.0){\rule[-0.200pt]{4.818pt}{0.400pt}}
\put(110.0,212.0){\rule[-0.200pt]{4.818pt}{0.400pt}}
\put(90,212){\makebox(0,0)[r]{ 2}}
\put(1419.0,212.0){\rule[-0.200pt]{4.818pt}{0.400pt}}
\put(110.0,292.0){\rule[-0.200pt]{4.818pt}{0.400pt}}
\put(90,292){\makebox(0,0)[r]{ 4}}
\put(1419.0,292.0){\rule[-0.200pt]{4.818pt}{0.400pt}}
\put(110.0,373.0){\rule[-0.200pt]{4.818pt}{0.400pt}}
\put(90,373){\makebox(0,0)[r]{ 6}}
\put(1419.0,373.0){\rule[-0.200pt]{4.818pt}{0.400pt}}
\put(110.0,454.0){\rule[-0.200pt]{4.818pt}{0.400pt}}
\put(90,454){\makebox(0,0)[r]{ 8}}
\put(1419.0,454.0){\rule[-0.200pt]{4.818pt}{0.400pt}}
\put(110.0,534.0){\rule[-0.200pt]{4.818pt}{0.400pt}}
\put(90,534){\makebox(0,0)[r]{ 10}}
\put(1419.0,534.0){\rule[-0.200pt]{4.818pt}{0.400pt}}
\put(110.0,615.0){\rule[-0.200pt]{4.818pt}{0.400pt}}
\put(90,615){\makebox(0,0)[r]{ 12}}
\put(1419.0,615.0){\rule[-0.200pt]{4.818pt}{0.400pt}}
\put(110.0,695.0){\rule[-0.200pt]{4.818pt}{0.400pt}}
\put(90,695){\makebox(0,0)[r]{ 14}}
\put(1419.0,695.0){\rule[-0.200pt]{4.818pt}{0.400pt}}
\put(110.0,776.0){\rule[-0.200pt]{4.818pt}{0.400pt}}
\put(90,776){\makebox(0,0)[r]{ 16}}
\put(1419.0,776.0){\rule[-0.200pt]{4.818pt}{0.400pt}}
\put(110.0,131.0){\rule[-0.200pt]{0.400pt}{4.818pt}}
\put(110,90){\makebox(0,0){ 0}}
\put(110.0,756.0){\rule[-0.200pt]{0.400pt}{4.818pt}}
\put(243.0,131.0){\rule[-0.200pt]{0.400pt}{4.818pt}}
\put(243,90){\makebox(0,0){ 10}}
\put(243.0,756.0){\rule[-0.200pt]{0.400pt}{4.818pt}}
\put(376.0,131.0){\rule[-0.200pt]{0.400pt}{4.818pt}}
\put(376,90){\makebox(0,0){ 20}}
\put(376.0,756.0){\rule[-0.200pt]{0.400pt}{4.818pt}}
\put(509.0,131.0){\rule[-0.200pt]{0.400pt}{4.818pt}}
\put(509,90){\makebox(0,0){ 30}}
\put(509.0,756.0){\rule[-0.200pt]{0.400pt}{4.818pt}}
\put(642.0,131.0){\rule[-0.200pt]{0.400pt}{4.818pt}}
\put(642,90){\makebox(0,0){ 40}}
\put(642.0,756.0){\rule[-0.200pt]{0.400pt}{4.818pt}}
\put(774.0,131.0){\rule[-0.200pt]{0.400pt}{4.818pt}}
\put(774,90){\makebox(0,0){ 50}}
\put(774.0,756.0){\rule[-0.200pt]{0.400pt}{4.818pt}}
\put(907.0,131.0){\rule[-0.200pt]{0.400pt}{4.818pt}}
\put(907,90){\makebox(0,0){ 60}}
\put(907.0,756.0){\rule[-0.200pt]{0.400pt}{4.818pt}}
\put(1040.0,131.0){\rule[-0.200pt]{0.400pt}{4.818pt}}
\put(1040,90){\makebox(0,0){ 70}}
\put(1040.0,756.0){\rule[-0.200pt]{0.400pt}{4.818pt}}
\put(1173.0,131.0){\rule[-0.200pt]{0.400pt}{4.818pt}}
\put(1173,90){\makebox(0,0){ 80}}
\put(1173.0,756.0){\rule[-0.200pt]{0.400pt}{4.818pt}}
\put(1306.0,131.0){\rule[-0.200pt]{0.400pt}{4.818pt}}
\put(1306,90){\makebox(0,0){ 90}}
\put(1306.0,756.0){\rule[-0.200pt]{0.400pt}{4.818pt}}
\put(1439.0,131.0){\rule[-0.200pt]{0.400pt}{4.818pt}}
\put(1439,90){\makebox(0,0){ 100}}
\put(1439.0,756.0){\rule[-0.200pt]{0.400pt}{4.818pt}}
\put(110.0,131.0){\rule[-0.200pt]{0.400pt}{155.380pt}}
\put(110.0,131.0){\rule[-0.200pt]{320.156pt}{0.400pt}}
\put(1439.0,131.0){\rule[-0.200pt]{0.400pt}{155.380pt}}
\put(110.0,776.0){\rule[-0.200pt]{320.156pt}{0.400pt}}
\put(774,29){\makebox(0,0){Generaciones}}
\put(774,838){\makebox(0,0){BACP 10 - Mejor resultado}}
\sbox{\plotpoint}{\rule[-0.400pt]{0.800pt}{0.800pt}}%
\sbox{\plotpoint}{\rule[-0.200pt]{0.400pt}{0.400pt}}%
\put(1279,736){\makebox(0,0)[r]{Mejor}}
\sbox{\plotpoint}{\rule[-0.400pt]{0.800pt}{0.800pt}}%
\put(1299.0,736.0){\rule[-0.400pt]{24.090pt}{0.800pt}}
\put(123,776){\usebox{\plotpoint}}
\multiput(124.41,752.16)(0.509,-3.658){21}{\rule{0.123pt}{5.743pt}}
\multiput(121.34,764.08)(14.000,-85.080){2}{\rule{0.800pt}{2.871pt}}
\multiput(204.41,670.00)(0.509,-1.278){19}{\rule{0.123pt}{2.169pt}}
\multiput(201.34,674.50)(13.000,-27.498){2}{\rule{0.800pt}{1.085pt}}
\put(137.0,679.0){\rule[-0.400pt]{15.899pt}{0.800pt}}
\multiput(403.41,634.78)(0.509,-1.789){21}{\rule{0.123pt}{2.943pt}}
\multiput(400.34,640.89)(14.000,-41.892){2}{\rule{0.800pt}{1.471pt}}
\put(216.0,647.0){\rule[-0.400pt]{44.807pt}{0.800pt}}
\multiput(443.41,590.34)(0.509,-1.217){21}{\rule{0.123pt}{2.086pt}}
\multiput(440.34,594.67)(14.000,-28.671){2}{\rule{0.800pt}{1.043pt}}
\put(416.0,599.0){\rule[-0.400pt]{6.263pt}{0.800pt}}
\multiput(483.41,544.73)(0.509,-3.263){19}{\rule{0.123pt}{5.123pt}}
\multiput(480.34,555.37)(13.000,-69.367){2}{\rule{0.800pt}{2.562pt}}
\put(456.0,566.0){\rule[-0.400pt]{6.263pt}{0.800pt}}
\multiput(523.41,477.00)(0.509,-1.278){19}{\rule{0.123pt}{2.169pt}}
\multiput(520.34,481.50)(13.000,-27.498){2}{\rule{0.800pt}{1.085pt}}
\put(495.0,486.0){\rule[-0.400pt]{6.504pt}{0.800pt}}
\multiput(696.41,432.48)(0.509,-3.304){19}{\rule{0.123pt}{5.185pt}}
\multiput(693.34,443.24)(13.000,-70.239){2}{\rule{0.800pt}{2.592pt}}
\put(535.0,454.0){\rule[-0.400pt]{38.544pt}{0.800pt}}
\put(708.0,373.0){\rule[-0.400pt]{176.098pt}{0.800pt}}
\sbox{\plotpoint}{\rule[-0.200pt]{0.400pt}{0.400pt}}%
\put(1279,695){\makebox(0,0)[r]{Castigos}}
\put(1299.0,695.0){\rule[-0.200pt]{24.090pt}{0.400pt}}
\put(123,131){\usebox{\plotpoint}}
\put(123.0,131.0){\rule[-0.200pt]{317.024pt}{0.400pt}}
\put(1279,654){\makebox(0,0)[r]{Max. carga}}
\multiput(1299,654)(20.756,0.000){5}{\usebox{\plotpoint}}
\put(1399,654){\usebox{\plotpoint}}
\put(123,776){\usebox{\plotpoint}}
\put(123.00,776.00){\usebox{\plotpoint}}
\put(143.76,776.00){\usebox{\plotpoint}}
\put(164.51,776.00){\usebox{\plotpoint}}
\put(185.27,776.00){\usebox{\plotpoint}}
\put(206.02,776.00){\usebox{\plotpoint}}
\put(226.78,776.00){\usebox{\plotpoint}}
\put(247.53,776.00){\usebox{\plotpoint}}
\put(268.29,776.00){\usebox{\plotpoint}}
\put(289.04,776.00){\usebox{\plotpoint}}
\put(309.80,776.00){\usebox{\plotpoint}}
\put(330.56,776.00){\usebox{\plotpoint}}
\put(351.31,776.00){\usebox{\plotpoint}}
\put(372.07,776.00){\usebox{\plotpoint}}
\put(392.82,776.00){\usebox{\plotpoint}}
\put(413.58,776.00){\usebox{\plotpoint}}
\put(434.33,776.00){\usebox{\plotpoint}}
\put(455.09,776.00){\usebox{\plotpoint}}
\put(475.84,776.00){\usebox{\plotpoint}}
\put(496.60,776.00){\usebox{\plotpoint}}
\put(517.35,776.00){\usebox{\plotpoint}}
\multiput(522,776)(6.415,-19.739){2}{\usebox{\plotpoint}}
\put(550.56,736.00){\usebox{\plotpoint}}
\put(571.32,736.00){\usebox{\plotpoint}}
\put(592.07,736.00){\usebox{\plotpoint}}
\put(612.83,736.00){\usebox{\plotpoint}}
\put(633.58,736.00){\usebox{\plotpoint}}
\put(654.34,736.00){\usebox{\plotpoint}}
\put(675.09,736.00){\usebox{\plotpoint}}
\multiput(695,736)(6.273,-19.785){3}{\usebox{\plotpoint}}
\put(728.10,695.00){\usebox{\plotpoint}}
\put(748.86,695.00){\usebox{\plotpoint}}
\put(769.62,695.00){\usebox{\plotpoint}}
\put(790.37,695.00){\usebox{\plotpoint}}
\put(811.13,695.00){\usebox{\plotpoint}}
\put(831.88,695.00){\usebox{\plotpoint}}
\put(852.64,695.00){\usebox{\plotpoint}}
\put(873.39,695.00){\usebox{\plotpoint}}
\put(894.15,695.00){\usebox{\plotpoint}}
\put(914.90,695.00){\usebox{\plotpoint}}
\put(935.66,695.00){\usebox{\plotpoint}}
\put(956.42,695.00){\usebox{\plotpoint}}
\put(977.17,695.00){\usebox{\plotpoint}}
\put(997.93,695.00){\usebox{\plotpoint}}
\put(1018.68,695.00){\usebox{\plotpoint}}
\put(1039.44,695.00){\usebox{\plotpoint}}
\put(1060.19,695.00){\usebox{\plotpoint}}
\put(1080.95,695.00){\usebox{\plotpoint}}
\put(1101.70,695.00){\usebox{\plotpoint}}
\put(1122.46,695.00){\usebox{\plotpoint}}
\put(1143.21,695.00){\usebox{\plotpoint}}
\put(1163.97,695.00){\usebox{\plotpoint}}
\put(1184.73,695.00){\usebox{\plotpoint}}
\put(1205.48,695.00){\usebox{\plotpoint}}
\put(1226.24,695.00){\usebox{\plotpoint}}
\put(1246.99,695.00){\usebox{\plotpoint}}
\put(1267.75,695.00){\usebox{\plotpoint}}
\put(1288.50,695.00){\usebox{\plotpoint}}
\put(1309.26,695.00){\usebox{\plotpoint}}
\put(1330.01,695.00){\usebox{\plotpoint}}
\put(1350.77,695.00){\usebox{\plotpoint}}
\put(1371.53,695.00){\usebox{\plotpoint}}
\put(1392.28,695.00){\usebox{\plotpoint}}
\put(1413.04,695.00){\usebox{\plotpoint}}
\put(1433.79,695.00){\usebox{\plotpoint}}
\put(1439,695){\usebox{\plotpoint}}
\put(110.0,131.0){\rule[-0.200pt]{0.400pt}{155.380pt}}
\put(110.0,131.0){\rule[-0.200pt]{320.156pt}{0.400pt}}
\put(1439.0,131.0){\rule[-0.200pt]{0.400pt}{155.380pt}}
\put(110.0,776.0){\rule[-0.200pt]{320.156pt}{0.400pt}}
\end{picture}

\end{center}
\begin{center}
% GNUPLOT: LaTeX picture
\setlength{\unitlength}{0.240900pt}
\ifx\plotpoint\undefined\newsavebox{\plotpoint}\fi
\sbox{\plotpoint}{\rule[-0.200pt]{0.400pt}{0.400pt}}%
\begin{picture}(1500,900)(0,0)
\sbox{\plotpoint}{\rule[-0.200pt]{0.400pt}{0.400pt}}%
\put(110.0,131.0){\rule[-0.200pt]{4.818pt}{0.400pt}}
\put(90,131){\makebox(0,0)[r]{ 0}}
\put(1419.0,131.0){\rule[-0.200pt]{4.818pt}{0.400pt}}
\put(110.0,260.0){\rule[-0.200pt]{4.818pt}{0.400pt}}
\put(90,260){\makebox(0,0)[r]{ 5}}
\put(1419.0,260.0){\rule[-0.200pt]{4.818pt}{0.400pt}}
\put(110.0,389.0){\rule[-0.200pt]{4.818pt}{0.400pt}}
\put(90,389){\makebox(0,0)[r]{ 10}}
\put(1419.0,389.0){\rule[-0.200pt]{4.818pt}{0.400pt}}
\put(110.0,518.0){\rule[-0.200pt]{4.818pt}{0.400pt}}
\put(90,518){\makebox(0,0)[r]{ 15}}
\put(1419.0,518.0){\rule[-0.200pt]{4.818pt}{0.400pt}}
\put(110.0,647.0){\rule[-0.200pt]{4.818pt}{0.400pt}}
\put(90,647){\makebox(0,0)[r]{ 20}}
\put(1419.0,647.0){\rule[-0.200pt]{4.818pt}{0.400pt}}
\put(110.0,776.0){\rule[-0.200pt]{4.818pt}{0.400pt}}
\put(90,776){\makebox(0,0)[r]{ 25}}
\put(1419.0,776.0){\rule[-0.200pt]{4.818pt}{0.400pt}}
\put(110.0,131.0){\rule[-0.200pt]{0.400pt}{4.818pt}}
\put(110,90){\makebox(0,0){ 0}}
\put(110.0,756.0){\rule[-0.200pt]{0.400pt}{4.818pt}}
\put(243.0,131.0){\rule[-0.200pt]{0.400pt}{4.818pt}}
\put(243,90){\makebox(0,0){ 10}}
\put(243.0,756.0){\rule[-0.200pt]{0.400pt}{4.818pt}}
\put(376.0,131.0){\rule[-0.200pt]{0.400pt}{4.818pt}}
\put(376,90){\makebox(0,0){ 20}}
\put(376.0,756.0){\rule[-0.200pt]{0.400pt}{4.818pt}}
\put(509.0,131.0){\rule[-0.200pt]{0.400pt}{4.818pt}}
\put(509,90){\makebox(0,0){ 30}}
\put(509.0,756.0){\rule[-0.200pt]{0.400pt}{4.818pt}}
\put(642.0,131.0){\rule[-0.200pt]{0.400pt}{4.818pt}}
\put(642,90){\makebox(0,0){ 40}}
\put(642.0,756.0){\rule[-0.200pt]{0.400pt}{4.818pt}}
\put(774.0,131.0){\rule[-0.200pt]{0.400pt}{4.818pt}}
\put(774,90){\makebox(0,0){ 50}}
\put(774.0,756.0){\rule[-0.200pt]{0.400pt}{4.818pt}}
\put(907.0,131.0){\rule[-0.200pt]{0.400pt}{4.818pt}}
\put(907,90){\makebox(0,0){ 60}}
\put(907.0,756.0){\rule[-0.200pt]{0.400pt}{4.818pt}}
\put(1040.0,131.0){\rule[-0.200pt]{0.400pt}{4.818pt}}
\put(1040,90){\makebox(0,0){ 70}}
\put(1040.0,756.0){\rule[-0.200pt]{0.400pt}{4.818pt}}
\put(1173.0,131.0){\rule[-0.200pt]{0.400pt}{4.818pt}}
\put(1173,90){\makebox(0,0){ 80}}
\put(1173.0,756.0){\rule[-0.200pt]{0.400pt}{4.818pt}}
\put(1306.0,131.0){\rule[-0.200pt]{0.400pt}{4.818pt}}
\put(1306,90){\makebox(0,0){ 90}}
\put(1306.0,756.0){\rule[-0.200pt]{0.400pt}{4.818pt}}
\put(1439.0,131.0){\rule[-0.200pt]{0.400pt}{4.818pt}}
\put(1439,90){\makebox(0,0){ 100}}
\put(1439.0,756.0){\rule[-0.200pt]{0.400pt}{4.818pt}}
\put(110.0,131.0){\rule[-0.200pt]{0.400pt}{155.380pt}}
\put(110.0,131.0){\rule[-0.200pt]{320.156pt}{0.400pt}}
\put(1439.0,131.0){\rule[-0.200pt]{0.400pt}{155.380pt}}
\put(110.0,776.0){\rule[-0.200pt]{320.156pt}{0.400pt}}
\put(774,29){\makebox(0,0){Generaciones}}
\put(774,838){\makebox(0,0){BACP 12 - Mejor resultado}}
\sbox{\plotpoint}{\rule[-0.400pt]{0.800pt}{0.800pt}}%
\sbox{\plotpoint}{\rule[-0.200pt]{0.400pt}{0.400pt}}%
\put(1279,736){\makebox(0,0)[r]{Mejor}}
\sbox{\plotpoint}{\rule[-0.400pt]{0.800pt}{0.800pt}}%
\put(1299.0,736.0){\rule[-0.400pt]{24.090pt}{0.800pt}}
\put(123,699){\usebox{\plotpoint}}
\multiput(177.41,661.40)(0.509,-5.870){21}{\rule{0.123pt}{9.057pt}}
\multiput(174.34,680.20)(14.000,-136.201){2}{\rule{0.800pt}{4.529pt}}
\put(123.0,699.0){\rule[-0.400pt]{12.768pt}{0.800pt}}
\multiput(310.41,530.84)(0.509,-1.942){21}{\rule{0.123pt}{3.171pt}}
\multiput(307.34,537.42)(14.000,-45.418){2}{\rule{0.800pt}{1.586pt}}
\put(190.0,544.0){\rule[-0.400pt]{28.667pt}{0.800pt}}
\multiput(417.41,478.14)(0.509,-2.063){19}{\rule{0.123pt}{3.338pt}}
\multiput(414.34,485.07)(13.000,-44.071){2}{\rule{0.800pt}{1.669pt}}
\put(323.0,492.0){\rule[-0.400pt]{22.404pt}{0.800pt}}
\multiput(682.41,427.84)(0.509,-1.942){21}{\rule{0.123pt}{3.171pt}}
\multiput(679.34,434.42)(14.000,-45.418){2}{\rule{0.800pt}{1.586pt}}
\put(429.0,441.0){\rule[-0.400pt]{60.707pt}{0.800pt}}
\multiput(922.41,374.89)(0.509,-2.105){19}{\rule{0.123pt}{3.400pt}}
\multiput(919.34,381.94)(13.000,-44.943){2}{\rule{0.800pt}{1.700pt}}
\put(695.0,389.0){\rule[-0.400pt]{54.443pt}{0.800pt}}
\put(934.0,337.0){\rule[-0.400pt]{121.654pt}{0.800pt}}
\sbox{\plotpoint}{\rule[-0.200pt]{0.400pt}{0.400pt}}%
\put(1279,695){\makebox(0,0)[r]{Castigos}}
\put(1299.0,695.0){\rule[-0.200pt]{24.090pt}{0.400pt}}
\put(123,131){\usebox{\plotpoint}}
\put(123.0,131.0){\rule[-0.200pt]{317.024pt}{0.400pt}}
\put(1279,654){\makebox(0,0)[r]{Max. carga}}
\multiput(1299,654)(20.756,0.000){5}{\usebox{\plotpoint}}
\put(1399,654){\usebox{\plotpoint}}
\put(123,621){\usebox{\plotpoint}}
\put(123.00,621.00){\usebox{\plotpoint}}
\put(143.76,621.00){\usebox{\plotpoint}}
\put(164.51,621.00){\usebox{\plotpoint}}
\put(185.27,621.00){\usebox{\plotpoint}}
\put(206.02,621.00){\usebox{\plotpoint}}
\put(226.78,621.00){\usebox{\plotpoint}}
\put(247.53,621.00){\usebox{\plotpoint}}
\put(268.29,621.00){\usebox{\plotpoint}}
\put(289.04,621.00){\usebox{\plotpoint}}
\multiput(309,621)(5.396,20.042){3}{\usebox{\plotpoint}}
\put(332.21,673.00){\usebox{\plotpoint}}
\put(352.97,673.00){\usebox{\plotpoint}}
\put(373.73,673.00){\usebox{\plotpoint}}
\put(394.48,673.00){\usebox{\plotpoint}}
\put(415.24,673.00){\usebox{\plotpoint}}
\multiput(416,673)(5.034,-20.136){2}{\usebox{\plotpoint}}
\put(436.90,621.00){\usebox{\plotpoint}}
\put(457.66,621.00){\usebox{\plotpoint}}
\put(478.41,621.00){\usebox{\plotpoint}}
\put(499.17,621.00){\usebox{\plotpoint}}
\put(519.92,621.00){\usebox{\plotpoint}}
\put(540.68,621.00){\usebox{\plotpoint}}
\put(561.44,621.00){\usebox{\plotpoint}}
\put(582.19,621.00){\usebox{\plotpoint}}
\put(602.95,621.00){\usebox{\plotpoint}}
\put(623.70,621.00){\usebox{\plotpoint}}
\put(644.46,621.00){\usebox{\plotpoint}}
\put(665.21,621.00){\usebox{\plotpoint}}
\put(685.97,621.00){\usebox{\plotpoint}}
\put(706.72,621.00){\usebox{\plotpoint}}
\put(727.48,621.00){\usebox{\plotpoint}}
\put(748.23,621.00){\usebox{\plotpoint}}
\put(768.99,621.00){\usebox{\plotpoint}}
\put(789.75,621.00){\usebox{\plotpoint}}
\put(810.50,621.00){\usebox{\plotpoint}}
\put(831.26,621.00){\usebox{\plotpoint}}
\put(852.01,621.00){\usebox{\plotpoint}}
\put(872.77,621.00){\usebox{\plotpoint}}
\put(893.52,621.00){\usebox{\plotpoint}}
\put(914.28,621.00){\usebox{\plotpoint}}
\put(935.03,621.00){\usebox{\plotpoint}}
\put(955.79,621.00){\usebox{\plotpoint}}
\put(976.55,621.00){\usebox{\plotpoint}}
\put(997.30,621.00){\usebox{\plotpoint}}
\put(1018.06,621.00){\usebox{\plotpoint}}
\put(1038.81,621.00){\usebox{\plotpoint}}
\put(1059.57,621.00){\usebox{\plotpoint}}
\put(1080.32,621.00){\usebox{\plotpoint}}
\put(1101.08,621.00){\usebox{\plotpoint}}
\put(1121.83,621.00){\usebox{\plotpoint}}
\put(1142.59,621.00){\usebox{\plotpoint}}
\put(1163.34,621.00){\usebox{\plotpoint}}
\put(1184.10,621.00){\usebox{\plotpoint}}
\put(1204.86,621.00){\usebox{\plotpoint}}
\put(1225.61,621.00){\usebox{\plotpoint}}
\put(1246.37,621.00){\usebox{\plotpoint}}
\put(1267.12,621.00){\usebox{\plotpoint}}
\put(1287.88,621.00){\usebox{\plotpoint}}
\put(1308.63,621.00){\usebox{\plotpoint}}
\put(1329.39,621.00){\usebox{\plotpoint}}
\put(1350.14,621.00){\usebox{\plotpoint}}
\put(1370.90,621.00){\usebox{\plotpoint}}
\put(1391.66,621.00){\usebox{\plotpoint}}
\put(1412.41,621.00){\usebox{\plotpoint}}
\put(1433.17,621.00){\usebox{\plotpoint}}
\put(1439,621){\usebox{\plotpoint}}
\put(110.0,131.0){\rule[-0.200pt]{0.400pt}{155.380pt}}
\put(110.0,131.0){\rule[-0.200pt]{320.156pt}{0.400pt}}
\put(1439.0,131.0){\rule[-0.200pt]{0.400pt}{155.380pt}}
\put(110.0,776.0){\rule[-0.200pt]{320.156pt}{0.400pt}}
\end{picture}

\end{center}

Como se puede apreciar, en los tres casos se obtuvieron soluciones factibles y
de altísima calidad.  A continuación se presenta un tabla que compara los resultados
obtenidos con los estudiados en el estado del arte donde se agrega además las
evaluaciones necesarias para encontrar el mejor resultado por primera vez y los
valores conocidos hasta el momento en el estudio del problema.


\begin{table}[h]
\centering
\caption{Analisis de resultados obtenidos}
\begin{tabular}{|c|c|c|p{2cm}|p{2cm}|p{2cm}|}
\hline 
Problema & Mejor F.O & Evaluaciones  & Carga máxima & Carga máxima óptima conocida & Máxima carga óptima teórica\tabularnewline
\hline 
\hline 
BACP 8 & 3.75 & 11438 & 17 & 17 & 16.625\tabularnewline
\hline 
BACP 10 & 6 & 31318 & 14 & 14 & 13.4\tabularnewline
\hline 
BACP 12 & 6 & 45767 & 18 & 18 & 17\tabularnewline
\hline
\end{tabular}
\end{table}

Finalmente se, vuelve a destacar, que el tiempo de ejecución no formo
parte de importante de los resultados por ser muy pequeño, siendo solo
perceptible durante la etapa de experimentación con problemas de 12
periodos o en factores de mutación elevados con desecho de soluciones
de calidad inferior.


\section{Conclusiones y trabajo futuro}
Muchos métodos y aproximaciones se han utilizado para atacar este
problema durante el tiempo. Permitiendo la obtención de resultados
óptimos con la utilización de métodos incompletos, en particular, con
búsquedas locales. Son estas ultimas las que más destacan a la hora de
extender el problema a situaciones más complejas como en los GBACP
donde existen aun más restricciones y consideraciones para generar una
solución.

Ahora si bien los métodos y las dimensiones del problema han cambiado
con el tiempo, el modelo matemático a permanecido con pocas
variaciones destacando solo el cambio en la función objetivo por otras
mas inteligentes, que aseguren de mejor manera el balance en la carga
académica de los alumnos.


En cuanto a la implementación del Algoritmo Genético, y luego de
analizar los resultados obtenidos y observar la evolución de la
población sobre el tiempo, es fácil notar que el alto grado de
restricción del problema hace difícil el trabajo meta heurística en
cuanto al mejoramiento de la población inicial.  Esto significa que el
resultado final esta, en la mayoría de los casos, íntimamente ligado a
la calidad de la población inicial puesto que una solución factible, por
lo general, no se aleja bruscamente del espacio soluciones óptimas.

Pese a lo anterior, se lograron obtener muy buenos resultados gracias
a la calidad resultante de la población inicial nacida de la
estrategia elegida para la generación de soluciones y al diseño del
operador mutación que se encargó de mover ramos de un semestre a otro
hasta encontrar un balance óptimo.

Otro resultado importante nace de la relación entre la función de
evaluación elegida por \cite{castro2001variable} y la elegida en esta
implementación. Se puedo apreciar como para un mismo valor de ``carga
máxima observada en algún periodo'' existía una colección de ``suma de
desviaciones con respecto a la media'' las cuales acusaban una mejoría
en el balance. Mejoría que, claramente, el trabajo mostrado
\cite{castro2001variable} no lograba detectar. Por esto último es que
se destaca la importancia del criterio presentado en
\cite{schaus2007deviation} para el cálculo de problemas de
optimización sujetos a restricciones que tienen como objetivo
presentar una solución balanceada.


También fue posible notar que para atacar de forma correcta este tipo
de problemas, se necesitan diseñar operadores que estén altamente
consiente de las restricciones, es decir, propagar restricciones de
manera adecuada y eficiente. Siguiendo este pensamiento es que se
vuelve interesante aplicar el conocimiento de otros campos como el de
la animación donde aplicaciones como Blender, 3D Studio o Unity poseen
herramientas para diseñar esqueletos con articulaciones las cuales al
ser sometidas a esfuerzos propagaran el efecto al resto de la
estructura acomodándola de tal manera que las resistencias de los
componentes no sean violadas. En este caso cada ramo perteneciente del
grafo de pre-requisitos se transformaría en una articulación y cada
valor del arreglo ``max period'' se transformaría en la resistencia
máxima a la deformación. Con esto en mente los operadores mutación y
cruzamiento se pueden diseñar como perturbaciones que modificarían
toda la estructura y la función de evaluación se transformaría en las
dimensiones de el tamaño de la caja que contengan dicho esqueleto y el
resto de los ramos sueltos. Lo mencionado anteriormente se perfila
como un  trabajo interesante en el busca de soluciones para este
tipo de problemas

\section{Bibliografía}
\bibliographystyle{plain}
\bibliography{Referencias}
\end{document} 
